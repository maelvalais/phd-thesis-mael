%%% CONFIGURATION

\usepackage{minitoc}
\usepackage{appendix}
\usepackage{blindtext}
\usepackage{longtable}
\usepackage{titlesec}
\usepackage[utf8]{inputenc}
\usepackage[french]{babel}
\usepackage[T1]{fontenc}
\usepackage{csquotes}
\usepackage[all]{xy}
\usepackage{lmodern}
\usepackage[backend=biber,style= alphabetic,date=long,maxbibnames=5,language=french]{biblatex}
\bibliography{biblio}

\usepackage{scalefnt}
\usepackage{booktabs} % for a nice table (\toprule, \bottomrule)
\usepackage{pdfpages}
\usepackage{amsthm}
\usepackage{graphicx}
\usepackage[intlimits]{amsmath}
%\usepackage{url} CLASHES WITH amsmath + intlimits
\usepackage{caption}
\usepackage{subcaption}
\usepackage{pgfplots}
\usepackage[european resistor, european voltage, european current]{circuitikz}
\usetikzlibrary{arrows,shapes,positioning}
\usetikzlibrary{decorations.markings,decorations.pathmorphing,decorations.pathreplacing}
\usetikzlibrary{calc,patterns,shapes.geometric}
\usepackage{rotating}
%\usepackage{slashbox}
\usepackage{times}
\usepackage{amssymb}
\usepackage{mathtools}
\usepackage{color}
\usepackage{makecell}
\usepackage{lastpage}
\usepackage{fancyhdr}
\usepackage{datetime}
\usepackage{eurosym}
\usepackage{algorithmic}
\usepackage[ruled,vlined]{algorithm2e}
\usepackage{latexsym}
\usepackage{lscape}
\usepackage{calc}
\usepackage{multirow}
\usepackage{array}
\usepackage{setspace}
\usepackage{enumitem}

\setstretch{1.3}

\renewcommand{\headrulewidth}{1pt}

\fancyhead[LE,RO]{\thepage}
\fancyhead[LO]{\leftmark }
\fancyhead[RE]{\rightmark }
\fancyfoot[LE,RO]{}
\fancyfoot[LO,C,RE]{}

\pagestyle{fancy}
\usepackage[includefoot,nomarginpar,
top=25mm,bottom=10mm,
head=5mm,headsep=7mm,
footskip=13mm,
hmargin=25mm,bindingoffset=10mm]{geometry}

\definecolor{couleurlien}{RGB}{0,0,101}
% \usepackage[
% pdftex,
% colorlinks,
% hyperindex,
% plainpages=false,
% bookmarksopen,
% bookmarksnumbered,
% pdfusetitle,
% linkcolor=couleurlien,
% citecolor=black,
% filecolor=magenta,
% urlcolor=cyan
% ]{hyperref}

\titleformat
{\chapter}
[display]
{}
{ \rule{\textwidth}{2pt}
\vspace{-5ex}
\centering
\MakeUppercase{\large{C}\normalsize{hapitre} \ \large\thechapter}}
%{0.05ex}
{0.5ex}
{
\rule{\textwidth}{0.5pt}
\centering
\bfseries\Large
}
[
\vspace{-2.3ex}
\rule{\textwidth}{2pt}
]
\setcounter{secnumdepth}{2}
\setcounter{tocdepth}{2}
\setcounter{minitocdepth}{1}
\setcounter{minitocdepth}{1}

\newcommand\hel[1]{{\color{green} #1}}
\newcommand\zei[1]{{\color{blue} #1}}

\renewcommand{\ref}[1]{\hypersetup{linkcolor=red}
\ref{#1}
\hypersetup{linkcolor=red}}

%%% MACROS


\newcommand{\satoulouse}{{\sc \texttt {SAToulouse}}}

\newcommand{\nameTool}{{\sc \texttt {TouIST}}}

\newcommand{\pre}[1]{\mathit{Pre}(#1)}
\newcommand{\cond}[1]{\mathit{Cond}(#1)}
\newcommand{\add}[1]{\mathit{Add}(#1)}
\newcommand{\del}[1]{\mathit{Del}(#1)}
\newcommand{\constr}[1]{\mathit{Constr}(#1)}
\newcommand{\events}[1]{\mathit{Events}(#1)}

\newcommand{\codrule}[1]{\textit{#1}}
\newcommand{\codage}[1]{$$#1$$}
\newcommand{\ifthen}[3]{\left(\begin{array}{l|l}{#1}\hookrightarrow{#2} & {#3}\end{array}\right)}
\newcommand{\biget}[1]{\stackrel{\bigwedge}{_{#1}}~}
\newcommand{\bigou}[1]{\stackrel{\bigvee}{_{#1}}~}
\newcommand{\causal}[1]{\stackrel{#1}{\to}}
%\newcommand{\txt}[1]{\mathrm{#1}}
\newcommand{\prop}[1]{\small\texttt{#1}}

%\newcommand{\bigexists}[1]{\stackrel{\exists}{_{#1}}~}
\newcommand\ScaleExists[1]{\vcenter{\hbox{\scalefont{#1}$\exists$}}}
\newcommand\ScaleForall[1]{\vcenter{\hbox{\scalefont{#1}$\forall$}}}

\DeclareMathOperator*\bigexists{%
  \vphantom\sum
  \mathchoice{\ScaleExists{2}}{\ScaleExists{1.4}}{\ScaleExists{1}}{\ScaleExists{0.75}}}
\DeclareMathOperator*\bigforall{%
  \vphantom\sum
  \mathchoice{\ScaleForall{2}}{\ScaleForall{1.4}}{\ScaleForall{1}}{\ScaleForall{0.75}}}

\newcommand{\suchthat}{\,\mid\,}

% Added by Mael
\newcommand {\Cond}[1] {\mathit{Pre}(#1)}
\newcommand {\Add}[1] {\mathit{Add}(#1)}
\newcommand {\Del}[1] {\mathit{Del}(#1)}
\renewcommand {\a} {a} % an action
\renewcommand {\b} {a'} % an other action
\renewcommand {\c} {a''} % yet an other action
\newcommand {\A} {\mathcal{A}} % the set of actions
\let\O\undefined % just to make sure I don't use \O inadvertedly
\newcommand {\F} {\mathcal{F}} % the set of fluents
\newcommand {\f} {f} % a fluent
\newcommand {\noop}[2] {\textit{noop}_{#1,#2}} % noop
\newcommand {\noopSingle}[1] {\textit{noop}_{#1}} % noop
\newcommand {\open}[2] {\textit{open}_{#1,#2}} % open
\newcommand {\openSingle}[1] {\textit{open}_{#1}} % open
\newcommand {\openSet} {\Delta} % open
%\newcommand {\depth} {\textit{depth}}
\newcommand {\depth} {k}
\newcommand {\length} {\textit{length}}
\newcommand {\X} {X} % set of all variables
\renewcommand {\S} {S} % one state
\newcommand {\I} {I} % initial set of fluents
\newcommand {\G} {G} % goal
\newcommand{\Bigforall}[1] {\bigwedge\limits_{\substack{#1}}}
\newcommand{\BigforallTwo}[2] {\Bigforall{#1}^{#2}}
\newcommand{\Bigexists}[1] {\bigvee\limits_{\substack{#1}}}
\newcommand{\BigexistsTwo}[2] {\Bigexists{#1}^{#2}}
\newcommand{\leftselect}[1] {\textit{left}(#1)}
\newcommand{\rightselect}[1] {\textit{right}(#1)}

\newcommand {\mael}[1] {{\color{cyan}[\undeline{Maël}: #1]}}
\newcommand {\fred}[1] {{\color{purple}[\underline{Fred}: #1]}}
\newtheorem{definition}{Définition}
\newcommand{\Definition}[1]{\begin{definition}#1\end{definition}}


%%% MACROS NIM

% Typo
\newcommand{\vs}{\emph{vs.}\@\xspace}
\newcommand{\touist}{\textsc{TouIST}\xspace}
\renewcommand{\satoulouse}{\textsc{SAToulouse}\xspace}
\newcommand{\guill}[1]{\emph{#1}}
\newcommand{\etc}{\emph{etc.}\@\xspace}
\newcommand{\warning}[1]{\textcolor{red}{#1}}

% sémantique
\newcommand{\interpret}[1][NIL]{%
    \ifthenelse{\equal{#1}{NIL}}{\mathscr{I}}{\mathscr{I}\Big ( #1 \Big )}%
}

% Logique
\newcommand{\limp}{\rightarrow}
\newcommand{\lequiv}{\leftrightarrow}
\newcommand{\IMPL}[0]{\longrightarrow}
\newcommand{\AND}[0]{\wedge}
\newcommand{\OR}[0]{\vee}
\newcommand{\NOT}[0]{\neg}
\newcommand{\FALSE}[0]{\perp}
\newcommand{\TRUE}[0]{\top}
\newcommand{\IFF}[0]{\leftrightarrow}
\newcommand{\ufb}[0]{\mbox{$\stackrel{?}{=}$}} % unifiable
\newcommand{\set}[1]{\{#1\}}
\newcommand{\atL}[2]{{\Large \geqslant}_{#1}^{#2}}
\newcommand{\atM}[2]{{\Large \leqslant}_{#1}^{#2}}
\newcommand{\exact}[2]{{\Large \#}_{#1}^{#2}}


% macro du \game
\newcommand{\game}{jeu de Nim\xspace}
\newcommand{\nbAllumettes}{\mathit{NA}}
\newcommand{\nbJoueurs}{\mathit{NJ}}
\newcommand{\matchesSet}{A}
\newcommand{\turnsSet}{T}
\renewcommand{\turn}[2][0]{\mathit{tour\_de\_}#1(#2)}
\newcommand{\rest}[2]{\mathit{reste}(#1, #2)}
\newcommand{\takes}[2][2]{\mathit{prend\_2}(#1)}
\newcommand{\lost}[1][0]{#1\_\mathit{perd}}

% MACROS TLP-GP

\newcommand{\Link}[3]{\mathit{Link}(#1,#2,#3)}
