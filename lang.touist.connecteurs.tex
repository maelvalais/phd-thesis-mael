\subsection{Les connecteurs généralisés : $\bigwedge_{i \in A}$ et $\bigvee_{i \in A}$}

Pour simplifier l'écriture de certaines formules, il est possible d'utiliser des connecteur logiques généralisés.
Par exemple, considérons une grille de \emph{carré latin}\footnote{comme un sudoku sans la contrainte sur les régions : 1 chiffre par case qui apparaît une seule fois par ligne et par colonne.} $4\times 4$ à résoudre :
\begin{center}
\texttt{
\begin{tabular}{|c|c|c|c|}
	\hline
	~ & ~ & 1 & ~ \\ \hline
	~ &~ & ~ & 2 \\ \hline
	4 & ~ & ~ & ~ \\ \hline
	~ & ~ & 3 &  \\ \hline
\end{tabular}
}\\
\end{center}

Pour chacune des cases, nous définissons quatre variables propositionnelles correspondant aux quatre valeurs possibles $\{1,2,3,4\}$ qu'on écrit aussi $[1..4]$. Ainsi, $p(i,j,k)$ représentera la proposition "La case de coordonnées $(i,j)$ contient la valeur $k$".\\

Pour imposer que la case de coordonnées (2,1) contient l'une des quatres valeurs possibles, nous pouvons utiliser la formule : \\
\[\texttt{p(2,1,\hl{1}) or p(2,1,\hl{2}) or p(2,1,\hl{3}) or p(2,1,\hl{4})}\]

Cette succession de $\vee$ ou seul l'un des indices (surligné) varie peut être condensée en utilisant le $\bigvee$ : 
\[\bigvee_{k\in [1..4]} \texttt{p(2,1,\hl{k})} \]
qui signifie donc "la case (2,1) contient l'une des quatre valeurs possibles", ou "pour l'une au moins des quatre valeurs possibles, cette valeur est dans la case (1,2)". \\

\noindent Remarquez l'analogie avec le $\Sigma$ mathématique avec lequel une expression telle que : 
\[f(x,1)+f(x,2)+\cdots+f(x,10)\]
serait condensée en :
\[\sum_{k\in[1..10]} f(x,k)\]


Nous voulons maintenant exprimer que \emph{toutes} les cases de la ligne 2 (cases de coordonnées (2,x)) contiennent l'une des quatre valeurs possibles : 

\[\bigvee_{k\in [1..4]} \texttt{p(2,\hl{1},k)} \wedge \bigvee_{k\in [1..4]} \texttt{p(2,\hl{2},k)}\wedge
\bigvee_{k\in [1..4]} \texttt{p(2,\hl{3},k)} \wedge
\bigvee_{k\in [1..4]} \texttt{p(2,\hl{4},k)}\] 

Pour cela, nous utilisons le $\bigwedge$ en faisant varier l'indice surligné et obtenons : 
\[\bigwedge_{j\in [1..4]}\bigvee_{k\in [1..4]} \texttt{p(2,\hl{j},k)}\]

Ce qui peut se lire "pour chaque case de la ligne 2, pour au moins une valeur de [1..4], cette valeur est dans la case". \\

Afin maintenant d'exprimer la même chose pour chaque ligne, nous généralisons en faisant varier le numéro de ligne : 

\[\bigwedge_{j\in [1..4]}\bigvee_{k\in [1..4]} \texttt{p(\hl{1},j,k)}\wedge \bigwedge_{j\in [1..4]}\bigvee_{k\in [1..4]} \texttt{p(\hl{2},j,k)}\wedge \bigwedge_{j\in [1..4]}\bigvee_{k\in [1..4]} \texttt{p(\hl{3},j,k)}\wedge \bigwedge_{j\in [1..4]}\bigvee_{k\in [1..4]} \texttt{p(\hl{4},j,k)}\]

soit, en utilisant à nouveau le $\bigwedge$ :

\[\bigwedge_{i\in [1..4]}\bigwedge_{j\in [1..4]}\bigvee_{k\in [1..4]} \texttt{p(\hl{i},j,k)}\]
Qui peut se lire "Pour chaque case (i,j) de la grille, il y a au moins une valeur parmi les quatre possibles qui y figure". \\

\noindent NB : sans ces facilités d'écriture, il aurait fallu 64 propositions et 63 connecteurs pour écrire la même chose. \\

Cette écriture est possible dans TouIST en utilisant des variables \texttt{\$i}, \texttt{\$j} et \texttt{\$k}.

\begin{center}
	\begin{tabular}{cV{4cm}} \toprule
		Logique propositionnelle & Langage TouIST \\ \midrule
		Pour $n\in \mathbb{N}^{*}$, $\underset{i\in [1..n]}{\bigwedge}p_{i}$ & 
		\begin{verbatim}
		bigand $i in [1..n]:
		    p($i)
		end
		\end{verbatim}
		\\ \hline
		Pour $n\in \mathbb{N}^{*}$, $\underset{i\in [1..n]}{\bigvee}p_{i}$ & 
		\begin{verbatim}
		bigor $i in [1..n]:
		    p($i)
		end
		\end{verbatim}
         \\ \hline
	\end{tabular}
\end{center}

Ce qui donne : 
\begin{verbatim}
1 |  bigand $i in [1..4]:
2 |    bigand $j in [1..4]:
3 |      bigor $k in [1..4]:
4 |        p($i,$j,$k)
5 |      end
6 |    end
7 |  end
\end{verbatim}

Exercice : Résoudre le Sudoku $4\times 4$ donné en exemple en utilisant les connecteurs généralisés :
\begin{itemize}
	\item Écrire une formule qui décrit l'état initial de la grille ci-dessus ;
    \item Recopier la formule qui impose que chaque cellule contienne \textbf{au moins} une valeur de $[1..4]$ ;
	\item Écrire une formule qui impose que chaque cellule contienne \textbf{au plus} une valeur parmi $[1..4]$ 
Indice : si une cellule contient une valeur alors elle n'en contient pas d'autre. Par exemple, pour exprimer que la case $(1,4)$ contient seulement un 3, il faut parvenir à écrire une formule disant que toute autre valeur que 3 est absente de la case $(2,3)$, en formule : 
\[\texttt{p(1,4,\hl{3}) => (not p(1,4,\hl{1}) and not p(1,4,\hl{2}) and not p(1,4,\hl{4})})\]
\end{itemize}
On remarque qu'en partie droite de cette implication, on retrouve toutes les valeurs possibles \emph{sauf} 3. Le langage de TouIST permet d'exprimer cela de manière compacte grâce à la clause \texttt{when}: 

\[\texttt{p(1,4,3) => bigand \$k in [1..4] \hl{when \$k!= 3}: not p(1,4,\$k)}\]

De manière générale dans : 
\begin{verbatim}
   ....
   bigand $i in [1..4] when $i != $j : ....
\end{verbatim}
(\$i ne prendra pas la valeur qu'a \$j)

Il ne reste plus qu'à appliquer ce principe à toutes les cases\ldots