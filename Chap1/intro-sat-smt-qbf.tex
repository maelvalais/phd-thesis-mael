%\subsubsection{SMT : SAT Modulo Theories}

Certains problèmes combinatoires nécessitent néanmoins de traiter des calculs sur les nombre naturels ou réels. Ceci peut être fait en utilisant seulement la logique propositionnelle (par exemple, $2+3=5$ pourrait être codé par $add_{2,3,5}$), mais c'est très inconfortable à moins qu'il n'y ait que quelques additions à faire. Ne parlons même pas des opérations de multiplication ou plus complexes. L'idée derrière la genèse de SMT a été de combiner des solveurs SAT avec un solveur arithmétique dans le but d'améliorer le traitement de la partie arithmétique du raisonnement. Dans de nombreux cas, ceci n'améliorera pas seulement l'efficacité du solveur, mais permettra aussi d'exprimer les contraintes arithmétiques des problèmes d'une manière radicalement plus compacte.

Pensez au jeu de Kamaji\footnote{\texttt{http://fr.wikipedia.org/wiki/Kamaji}}  où le joueur doit grouper des nombres adjacents dans une grille de sorte que leur somme soit égale à un nombre fixe. Résoudre le jeu nécessite essentiellement un raisonnement logique mais a aussi besoin d'un peu d'arithmétique (addition).


Si $x_{i,j}$ pour chaque case $(i,j)$ est un entier et $G(i,j,i,k)$ représente le fait que les cases $(i,j)$ à $(i,k)$ de la ligne $i$ forment un groupe, la contrainte de somme pourrait être exprimée par :
$$\sum_{m\in E}x_{i,m}=N$$
où $N$ est le nombre fixe et $E$ est $\{j,j+1,\ldots,k\}$. La logique propositionnelle pure n'est certainement pas adaptée pour de telles phrases !