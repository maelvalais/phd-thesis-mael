\section{Historique de \touist}\label{chap:touist:historique}

\subsection{Genèse de \satoulouse}

Lors de la conférence ICTTL'2011, %l'équipe 
(\citeauthor{GaScSt2011}) avaient présenté l'outil \satoulouse \cite{GaScSt2011}, dédié à la logique propositionnelle, dont les principales fonctions étaient :
\begin{enumerate}
    \item d'offrir un langage logique de haut niveau pour exprimer succinctement des formules complexes ;
    \item de trouver des modèles à ces formules en utilisant un solveur SAT performant.
\end{enumerate}

Bien sûr, il existait de nombreux outils logiques comme des prouveurs, assistants de preuves, éditeurs de tables de vérité\ldots disponibles sur Internet ; PROLOG aurait même pu être utilisé, mais aucun de ces outils ne correspondait aux exigences formulées, à savoir :
\begin{itemize}
\item l'outil doit être très facile à installer et à utiliser, sans syntaxe complexe ;
\item le prouveur peut être utilisé comme une boîte noire sans savoir comment il fonctionne ;
\item  aucune mise en forme normale,  aucun ordonnancement de clauses, ou aucune coupure PROLOG ne doivent être requis ;
\item seulement une petite connaissance en logique devrait être nécessaire.
\end{itemize} 

%L'équipe ne 
Ne trouvant aucun outil existant satisfaisant ces exigences, il a été décidé de développer un nouvel logiciel en 2010. L'objectif était de développer une interface qui permettrait d'utiliser confortablement un prouveur SAT (à savoir SAT4J \cite{DBLP:journals/jsat/BerreP10}). L'outil, appelé \satoulouse, est décrit dans \cite{GaScSt2011}. À travers cet outil, l'utilisateur pouvait expérimenter par lui-même l'idée qu'un langage logique n'est pas seulement descriptif mais peut aussi conduire à des calculs qui résolvent des problèmes de la vie réelle. En particulier, dans le domaine éducatif, \satoulouse donnait l'occasion lors de cours avec des étudiants de L1 et L2 de résoudre des Sudokus en quelques lignes, ainsi que beaucoup d'autres problèmes combinatoires (emplois du temps, coloration de carte, circuits électroniques\ldots).\

Voici les principales facilités qu'offraient \satoulouse :
\begin{itemize}
\item les formules entrées n'ont pas besoin d'être sous forme clausale et des connecteurs arbitraires peuvent être utilisés, la mise sous forme normale est faite dynamiquement pendant la saisie au clavier de l'utilisateur ;
\item des facilités d'utilisation de grandes conjonctions ou disjonctions sont offertes comme dans :
\end{itemize}
  \[\bigwedge_{i\in\{1..9\}}
  \bigvee_{j\in\{1..9\}}\bigwedge_{n\in\{1..9\}}\bigwedge_{m\in\{1..9\},m\neq
    n}(p_{i,j,n}\rightarrow \lnot p_{i,j,m})\]
\begin{itemize}
\item démarrer le solveur consiste à cliquer sur un bouton ;
\item l'outil affiche un modèle dans la syntaxe de la formule entrée.
\end{itemize}
Ainsi, il est possible de mettre à disposition de l'utilisateur la puissance de la logique propositionnelle avec pour seul pré-requis la formalisation de phrases en logique ainsi que les notions de validité et satisfiabilité pour résoudre, par exemple, des Sudokus.\

L'avantage de ce type d'outil tient au fait qu'un même solveur SAT peut être utilisé pour résoudre de nombreux autres problèmes combinatoires aussi facilement que pour le Sudoku : ils suffit de formaliser les contraintes. Ces problèmes vont de la gestion des emplois du temps, la coloration de cartes ou même le stockage de produits chimiques qui doivent être stockés dans des salles identiques/contiguës/non-contiguës en fonction de leur degré de compatibilité.

\subsection{Limites de \satoulouse et genèse de \touist}
Mais pendant ces années, nous avons remarqué quelques limitations dommageables de \satoulouse : de nombreux bugs, des défauts dans l'interface, le manque de modularité (si l'on souhaite changer le prouveur SAT utilisé), et l'ambiguïté et les limites de son langage.

Par exemple, des problèmes impliquant des contraintes de cardinalité, comme les règles du jeu de Takuzu\footnote{Connu aussi sous le nom de Binero.\\ \url{http://fr.wikipedia.org/wiki/Takuzu}} qui nécessitent de compter des 0 et des 1, ne peuvent être facilement formalisés : il manque des fonctionnalités permettant d'exprimer des choses comme \enquote{exactement 5 parmi 10 propositions sont vraies}. 

De plus, \satoulouse n'offre pas la possibilité de parcourir l'ensemble des modèles fournis par le solveur, il en retourne seulement un.



Les leçons tirées de trois années d'utilisation de \satoulouse pour l'enseignement de la logique sont que beaucoup des étudiants en informatique ont clairement pris conscience que la logique avait des applications réelles en ce qui concerne la résolution de problèmes, et beaucoup d'entre-eux ont acquis une capacité dans la formalisation de problèmes. Mais les défauts de \satoulouse rendent le déboggage vraiment difficile, d'une part parce qu'un seul modèle est affiché et en raison de la façon dont ce modèle est affiché, et d'autre part à cause des faibles capacités d'édition dont il dispose. En outre, seuls des problèmes combinatoires purs peuvent être traités, ce qui limite lourdement la prétention de résolution d'une large gamme de problèmes par \satoulouse concernant les problèmes du monde réel.

Un autre inconvénient de \satoulouse, pas nécessairement lié à l'enseignement de la logique, est son incapacité à être utilisé à partir de la ligne de commande : les travaux de recherche nécessitent d'automatiser l'utilisation des outils pour les utiliser dans le cadre de benchmarks.

Enfin, l'extension à des théories plus riches est également quelque chose qui peut intéresser les chercheurs, les ingénieurs ou les étudiants de cycles supérieurs. \satoulouse n'est certainement pas adapté pour la satisfiabilité modulo théories ou pour résoudre des problèmes de planification alors que la même architecture logicielle pourrait être utilisée en changeant juste le solveur.


Nous avons donc décidé de développer un nouveau logiciel qui comblerait ces lacunes et remplirait nos attentes. Nous l'avons appelé \touist qui signifie TOUlouse Integrated Satisfiability Tool et doit être prononcé \enquote{twist}.
 \touist est à la disposition du public pour téléchargement à partir du site suivant :
\begin{center}\url{ https://www.irit.fr/touist/ }\end{center}


Pour résumer, voici quelques fonctionnalités offertes par \touist et que \satoulouse ne propose pas :
\begin{itemize}
\item définition d'ensembles de domaines : $\bigwedge_{i\in A}$ vs. $\bigwedge_{i\in\{Paris,London,Roma,Madrid\}}$
\item liaisons multiples sur les indices: $\bigwedge_{i\in A,j\in B}$ vs. $\bigwedge_{i\in \cdots} \bigwedge_{j\in \cdots}$
\item calculs riches sur les indices ainsi que sur les ensembles de domaines: $\bigwedge_{i\in (A\cup (B \cap C))}$
\item primitives de contraintes de cardinalité intégrées: \enquote{auMoins} (resp. \enquote{auPlus}, \enquote{exactement}) \emph{tant} de valeurs sont vraies parmi \emph{ces valeurs}
\item les prédicats peuvent également être des variables définies sur des ensembles de domaines: $\bigwedge_{X\in \{A,B\},i\in \{1,2\}} X(i)$ vs. $\bigwedge_{i\in \{1,2\}} (A(i)\AND B(i))$
\item littéraux spéciaux définissant des contraintes entre nombres entiers ou réels : ($x+y\leq z$)
\item parcours facile des modèles successivement calculés par les solveurs
\item expressions régulières permettant un filtrage des littéraux pertinents
\item possibilité d'utiliser le logiciel en ligne de commande et/ou batch
\item nombreuses fonctionnalités d'édition et améliorations
\end{itemize}
