\subsection{Arithmétique linéaire avec SMT : le jeu de Takuzu}\label{chap:touist:touist:smt}

Le Takuzu (aussi appelé Binairo) est un jeu qui consiste à remplir une grille par déduction, de façon similaire au sudoku.
%\fred{On présente le codage SMT, mais si on voulait l'écrire en se limitant à SAT, on utiliserait "exact" mais ceci mènerait à une explosion combinatoire \mael{à vérifier}. Ici, cela nous permet d'avoir un codage (calculer la complexité en espace).}
Il est possible de le modéliser en utilisant un codage SAT mais, afin d'obtenir un codage plus compact, nous pouvons utiliser un codage SMT avec des atomes de QF-LIA (arithmétique linéaire sur les nombres entiers). Notons que \touist permet également d'utiliser SMT avec les logiques QF-IDL, QF-RDL (différences logiques sur les nombres entiers/rationnels) et QF-RDL (artihmétique linéaire sur les nombres rationnels). %Il est possible de se référer au manuel utilisateur (Chapitre~\ref{chap:reference-manual}, page~\pageref{chap:reference-manual}) pour davantage de détail sur l'utilisation de ces logiques.


Chaque grille de Takuzu, qui peut éventuellement avoir un nombre différent de lignes ($N_L$) et de colonnes ($N_C$), ne contient que des 0 et des 1.
Si l'on considère que $x_{i,j}\in\{0,1\}$ représente le nombre contenu dans la cellule de coordonnées $(i,j)$, cette contrainte peut se coder de la façon suivante :
\[
\BigforallTwo{i=1}{N_L} \BigforallTwo{j=1}{N_C} \left( (x_{i,j}=0) \lor (x_{i,j}=1) \right)
\]

\noindent La grille doit être complétée en respectant les trois règles suivantes :

\begin{itemize}
\item il doit y avoir autant de 0 et de 1 sur chaque ligne et sur chaque colonne :
\[
\BigforallTwo{i=1}{N_L} \left( \sum\limits_{j=1}^{N_C} x_{i,j} = \frac{N_C}{2} \right)
\land \BigforallTwo{j=1}{N_C} \left( \sum\limits_{i=1}^{N_L} x_{i,j} = \frac{N_L}{2} \right)
\]
\item il ne doit pas y avoir plus de deux chiffres identiques côte à côte :
\[
\BigforallTwo{i=1}{N_L} \BigforallTwo{j=1}{N_C-2} \left( \left( \BigexistsTwo{k=0}{2} (x_{i,j+k}\neq 0)\right) \land \left( \BigexistsTwo{k=0}{2} (x_{i,j+k}\neq 1)\right) \right)
\]
\[
\BigforallTwo{j=1}{N_C} \BigforallTwo{i=1}{N_L-2} \left( \left( \BigexistsTwo{k=0}{2} (x_{i+k,j}\neq 0)\right) \land \left( \BigexistsTwo{k=0}{2} (x_{i+k,j}\neq 1)\right) \right)
\]
\item il ne doit pas y avoir deux lignes ou deux colonnes identiques :
\[
\left( \BigforallTwo{i_1=1}{N_L} \BigforallTwo{\substack{i_2=1\\i_1\neq i_2}}{N_L} \BigexistsTwo{j=1}{N_C} (x_{i_1,j}\neq x_{i_2,j}) \right)
\land \left( \BigforallTwo{j_1=1}{N_C} \BigforallTwo{\substack{j_2=1\\j_1\neq j_2}}{N_C} \BigexistsTwo{i=1}{N_L} (x_{i,j_1}\neq x_{i,j_2}) \right)
\]
\end{itemize}

Nous présentons maintenant un exemple de jeu de Takuzu sur une grille $4\times4$ :


%\[
%\BigforallTwo{i=1}{4} \BigforallTwo{j=1}{4} \Bigforall{v\in x_{i,j}} x_{i,j}=v
%\]



\begin{center}
\texttt{
\begin{tabular}{|c|c|c|c|}
	\hline
	~ & 1 & ~ & 0 \\ \hline
	~ & ~ & 0 & ~ \\ \hline
	~ & 0 & ~ & ~ \\ \hline
	1 & 1 & ~ & 0 \\ \hline
\end{tabular}
}\\
\end{center}

Le codage \touist pour la résolution de cette grille est détaillé figure~\ref{fig:takuzu4x4-touist} et le modèle retourné, ainsi que la grille correspondante sont donnés figure~\ref{fig:modele-touist-cmd-line}.

%\[
%\bigwedge\limits_{\substack{\mathbf{i},\mathbf{j},\mathbf{value}\in [1..4],[1..4],\mathbf{x}_{\mathbf{i},\mathbf{j}}}}\left(x_{\mathbf{i},\mathbf{j}} = \mathbf{value}\right)\\
%\bigwedge\limits_{\substack{\mathbf{i},\mathbf{j}\in \mathbf{N},\mathbf{N}}}\left(\left(x_{\mathbf{i},\mathbf{j}} = 0\right) \vee \left(x_{\mathbf{i},\mathbf{j}} = 1\right)\right)\\
%\bigwedge\limits_{\substack{\mathbf{i}\in \mathbf{N}}}\left(\left(x_{\mathbf{i},1} + x_{\mathbf{i},2} + x_{\mathbf{i},3} + x_{\mathbf{i},4} = 2\right) \wedge \left(x_{1,\mathbf{i}} + x_{2,\mathbf{i}} + x_{3,\mathbf{i}} + x_{4,\mathbf{i}} = 2\right)\right)\\
%\bigwedge\limits_{\substack{\mathbf{i},\mathbf{j}\in \mathbf{N\_2},\mathbf{N}}}\left(\left(\neg \left(\left(x_{\mathbf{i},\mathbf{j}} = 0\right) \wedge \left(x_{\mathbf{i} + 1,\mathbf{j}} = 0\right) \wedge \left(x_{\mathbf{i} + 2,\mathbf{j}} = 0\right)\right)\right) \wedge \left(\neg \left(\left(x_{\mathbf{i},\mathbf{j}} = 1\right) \wedge \left(x_{\mathbf{i} + 1,\mathbf{j}} = 1\right) \wedge \left(x_{\mathbf{i} + 2,\mathbf{j}} = 1\right)\right)\right) \wedge \left(\neg \left(\left(x_{\mathbf{j},\mathbf{i}} = 0\right) \wedge \left(x_{\mathbf{j},\mathbf{i} + 1} = 0\right) \wedge \left(x_{\mathbf{j},\mathbf{i} + 2} = 0\right)\right)\right) \wedge \left(\neg \left(\left(x_{\mathbf{j},\mathbf{i}} = 1\right) \wedge \left(x_{\mathbf{j},\mathbf{i} + 1} = 1\right) \wedge \left(x_{\mathbf{j},\mathbf{i} + 2} = 1\right)\right)\right)\right)\\
%\bigwedge\limits_{\substack{\mathbf{i},\mathbf{j}\in \mathbf{N},\mathbf{N}\\ \mathbf{i} \neq \mathbf{j}}}\left(\bigvee\limits_{\substack{\mathbf{k}\in \mathbf{N}}}\left(x_{\mathbf{i},\mathbf{k}} \neq x_{\mathbf{j},\mathbf{k}}\right) \wedge \bigvee\limits_{\substack{\mathbf{k}\in \mathbf{N}}}\left(x_{\mathbf{k},\mathbf{i}} \neq x_{\mathbf{k},\mathbf{j}}\right)\right)
%\]


\begin{figure}
\begin{lstlisting}
;; SMT2 logic used: QF_LIA
$n=4
$N=[1..$n]
$N_2=[1..$n-2]

;; Define the initial grid here
$x(1,1)=[ ]	$x(1,2)=[1]	$x(1,3)=[ ]	$x(1,4)=[0]
$x(2,1)=[ ]	$x(2,2)=[ ]	$x(2,3)=[0]	$x(2,4)=[ ]
$x(3,1)=[ ]	$x(3,2)=[0]	$x(3,3)=[ ]	$x(3,4)=[ ]
$x(4,1)=[1]	$x(4,2)=[1]	$x(4,3)=[ ]	$x(4,4)=[0]

;; Initialize values with defined sets
bigand $i,$j,$value in [1..4],[1..4],$x($i,$j):
  x($i,$j)==$value
end

;; Each cell contains either 0 or 1
bigand $i,$j in $N,$N:
  (x($i,$j)==0) or (x($i,$j)==1)
end

;; There is an equal number of 1s and 0s in each row and column
bigand $i in $N:
  (x($i,1)+x($i,2)+x($i,3)+x($i,4)==2)
  and (x(1,$i)+x(2,$i)+x(3,$i)+x(4,$i)==2)
end

;; There is no more than two of either number adjacent to each other
bigand $i,$j in $N_2,$N:
      (not ((x($i,$j)==0) and (x($i+1,$j)==0) and (x($i+2,$j)==0)))
  and (not ((x($i,$j)==1) and (x($i+1,$j)==1) and (x($i+2,$j)==1)))
  and (not ((x($j,$i)==0) and (x($j,$i+1)==0) and (x($j,$i+2)==0)))
  and (not ((x($j,$i)==1) and (x($j,$i+1)==1) and (x($j,$i+2)==1)))
end

;; There can be no identical rows or columns
bigand $i,$j in $N,$N when $i!=$j:
    bigor $k in $N:
      (x($i,$k)!=x($j,$k))
    end
  and
    bigor $k in $N:
      (x($k,$i)!=x($k,$j))
    end
end
\end{lstlisting}
\caption{Codage d'un Takuzu $4\times 4$ en \touist.}% À droite, la grille à résoudre.}
    \label{fig:takuzu4x4-touist}
\end{figure}


\begin{figure}
\begin{minipage}[c]{.60\linewidth}
\begin{center}
\texttt{
\begin{tabular}{|c|c|c|c|}
	\hline
	0 & 1 & 1 & 0 \\ \hline
	1 & 0 & 0 & 1 \\ \hline
	0 & 0 & 1 & 1 \\ \hline
	1 & 1 & 0 & 0 \\ \hline
\end{tabular}}\\
\end{center}
\end{minipage} \hfill
\begin{minipage}[c]{.38\linewidth}
\begin{lstlisting}
0 x(1,1)
0 x(1,4)
0 x(2,2)
0 x(2,3)
0 x(3,1)
0 x(3,2)
0 x(4,3)
0 x(4,4)
1 x(1,2)
1 x(1,3)
1 x(2,1)
1 x(2,4)
1 x(3,3)
1 x(3,4)
1 x(4,1)
1 x(4,2)
\end{lstlisting}
\end{minipage}
\caption{Grille remplie correspondant au modèle retourné par \touist. À droite, le modèle tel que retourné par \texttt{touist} en ligne de commande.}
    \label{fig:modele-touist-cmd-line}
\end{figure}

