%
% SOLVEURS INTEGRES
%
\subsection{Solveurs intégrés}


%
% SOLVEURS EXTERNES
%
\subsection{Intégration avec des solveurs externes}

% \noindent La plupart des solveurs SAT et QBF acceptent le format standard~\href{http://www.satcompetition.org/2009/format-benchmarks2009.html}{DIMACS} (resp.
% QDIMACS) comme langage d'entrée. Il suffit de donner la sortie DIMACS de \touist directement à un solveur et d'utiliser la table de correspondance (voir Section~\ref{mapping-table}).
% Mais il est également possible d'utiliser \touist pour effectuer à la fois l'appel d'un solveur externe et la traduction inverse du modèle DIMACS retourné par ce dernier vers les noms de variables propositionnelles en utilisant l'argument \mdcode{--solver}:%mdk
% \begin{footnotesize}
% \begin{mdpre}%mdk
% \noindent\preindent{4}touist~{}[--sat\textbar{}--qbf]~--solver="\textless{}cmd-and-arguments\textgreater{}"~{}[--verbose]%mdk
% \end{mdpre}
% \end{footnotesize}
% \noindent A des fins de débogage, il est possible d'ajouter \mdcode{--verbose} pour afficher les entrée et sorties stdin/stdout/stderr.
% Le code de sortie de \touist sera alors le même qu'en utilisant \mdcode{--solve}.

% Les solveurs externes doivent utiliser les conventions Minisat + (Q)DIMACS suivantes :
% \begin{itemize}
%     \item accepter DIMACS ou QDIMACS sur l'entrée standard (stdin) ;
%     \item afficher un modèle (ou un modèle partiel) au format DIMACS sur la sortie standard (stdout) ; le modèle peut éventuellement être étendu sur plusieurs lignes, chacune commençant par \mdcode{v}, \mdcode{V} ou rien (pour la compatibilité avec Minisat), et chaque ligne peut de manière optionnelle se terminer par 0.
% \begin{footnotesize}
% \begin{mdpre}%mdk
% \noindent\preindent{2}v~-1~2~-3~4~0\\
% \preindent{2}v~5~-6~0%mdk
% \end{mdpre}
% \end{footnotesize}
% %\begin{itemize}[noitemsep,topsep=\mdcompacttopsep]%mdk

% \item retourner 10 (comme code d'erreur) sur le problème est SAT, 20 s'il est UNSAT.%mdk
% %mdk
% \end{itemize}%mdk


% \noindent Les solveurs SAT que nous avons testés (\mdcode{brew} est disponible pour~\href{http://linuxbrew.sh}{linux} et~\href{https://brew.sh}{mac}):%mdk

% \begin{itemize}%mdk

% \item{}
% \href{http://minisat.se}{minisat}%mdk
% \begin{footnotesize}
% \begin{mdpre}%mdk
% \noindent brew~install~minisat\\
% touist~test/sat/sudoku.touist~--solver="minisat~/dev/stdin~/dev/stdout"%mdk
% \end{mdpre}%mdk
% \end{footnotesize}

% \item{}
% \href{http://fmv.jku.at/picosat}{picosat} (2015, version 965, SAT Race'15)%mdk
% \begin{footnotesize}
% \begin{mdpre}%mdk
% \noindent brew~install~touist/touist/picosat\\
% touist~test/sat/sudoku.touist~--solver="picosat~--partial"%mdk
% \end{mdpre}%mdk
% \end{footnotesize}

% \item{}
% \href{https://www.labri.fr/perso/lsimon/glucose}{glucose} (2016, version 4.1, syrup est la version parallèle)%mdk
% \begin{footnotesize}
% \begin{mdpre}%mdk
% \noindent brew~install~touist/touist/glucose\\
% touist~test/sat/sudoku.touist~--solver="glucose~-model"\\
% touist~test/sat/sudoku.touist~--solver="glucose-syrup~-model"%mdk
% \end{mdpre}%mdk
% \end{footnotesize}%mdk
% \end{itemize}%mdk

% \noindent Les solveurs QBF que nous avons testés:%mdk

% \begin{itemize}%mdk

% \item{}
% \href{https://www.react.uni-saarland.de/tools/caqe/index.html}{caqe} (2017-07-19, CAQE qbfeval 2017, version binaire sans certification).
% Download the version \emph{CAQE qbfeval 2017 (2017-07-19) binary release without certification}
% which will give you \mdcode{caqe-mac}:%mdk
% \begin{footnotesize}
% \begin{mdpre}%mdk
% \noindent touist~test/qbf/allumettes2.touist~--qbf
%     --solver="./caqe-mac~--partial-assignments"%mdk
% \end{mdpre}
% \end{footnotesize}

% They also have a Homebrew tap repository but this version does not contain the needed \mdcode{--partial-assignments}.%mdk%mdk

% \item{}
% \href{https://github.com/perebor/qute}{qute} (2017-02-27, fork maelvalais/qute, minisat-based)%mdk
% \begin{footnotesize}
% \begin{mdpre}%mdk
% \noindent brew~install~touist/touist/qute\\
% touist~test/qbf/allumettes2.touist~--qbf\\
% \qquad\qquad --solver="qute~--partial-certificate"%mdk
% \end{mdpre}%mdk
% \end{footnotesize}

% \item{}
% \href{http://lonsing.github.io/depqbf/}{depqbf} (2017-08-02, DepQBF 6.03, Minisat-based QCDCL)%mdk
% \begin{mdpre}%mdk
% \noindent brew~install~depqbf\\
% touist~test/qbf/allumettes2.touist~--qbf
% \qquad\qquad --solver="depqbf~--qdo~--no-dynamic-nenofex"%mdk
% \end{mdpre}%mdk

% \item{}
% \href{http://fmv.jku.at/quantor/}{quantor} (2014-10-26, Quantor 3.2). It is not necessary to use this
% solver externally as it is included with \mdcode{touist} (see Section~\ref{usage-qbf-solver}).%mdk
% \begin{mdpre}%mdk
% \noindent brew~install~touist/touist/quantor\\
% touist~test/qbf/allumettes2.touist~--qbf~--solver="quantor"%mdk
% \end{mdpre}%mdk

% \item{}
% \href{http://sat.inesc-id.pt/~mikolas/sw/areqs/}{rareqs} (2012, v1.1, CEGAR)%mdk
% \begin{mdpre}%mdk
% \noindent brew~install~touist/touist/rareqs\\
% touist~test/qbf/allumettes2.touist~--qbf~--solver="rareqs"%mdk
% \end{mdpre}%mdk
% %mdk
% \end{itemize}%mdk


%
% AUTRES SORTIES
%
\subsection{Autres sorties (\LaTeX...)}
