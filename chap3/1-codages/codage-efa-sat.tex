%
% codage SAT-EFA
%
%\subsection{Codage SAT de référence dans les espaces d'états}

\fred{Présenter le codage KS92,95 SAT-EFA classique}

\fred{Attention, ici copier/coller ouvrage SAT 2008 Maris, Régnier, Vidal}
{\color{red}

\subsubsection{Codage dans les espaces d'états avec frame-axiomes explicatifs}

\begin{figure}\label{steps:sat}
\begin{footnotesize}
%(a)\\[1em]

\fred{Rajouter la figure que j'ai enlevée ici !!!}
%   \xymatrix@C=0.1pc@R=1pc{
%   \text{S}_{0} (\textit{Init}) \ar@{->}[r] & \fbox{$x_{1}\equiv$ S$_{1}$} \ar@{>}[r]  & \fbox{$x_{2}\equiv$ S$_{2}$} \ar@{>}[r] & \fbox{$x_{3}\equiv$ S$_{3}$} \ar@{>}[r] & \fbox{$x_{4}\equiv$ S$_{4}$}
%   \ar@{>}[r] & \fbox{$x_{5}\equiv$ S$_{5}$} \ar@{>}[r] & \fbox{$x_{6}\equiv$ S$_{6}$} \ar@{>}[r] & \fbox{$x_{7}\equiv$ S$_{7}$} \ar@{>}[r] & \text{S}_{8} (\textit{Goal})
%   }
\end{footnotesize}
\vspace{1em}
\caption{Transitions of an 8-step plan in SAT/SMT encoding}
\end{figure}


Les codages dans les espaces d'états sont basés sur les transitions entre les étapes successives du plan en partant de l'état initial pour arriver au but.
%Le parallélisme y est codé grâce à la notion d'indépendance entre actions simultanées. 
Pour conserver les fluents non affectés par les actions qui doivent être exécutées dans une étape du plan, on va coder la notion de frame-axiome.
Nous décrivons d'abord la technique la plus efficace en termes de compacité et
de temps de résolution \cite{MK99} qui utilise des frame-axiomes explicatifs.
Nous présentons ensuite une variation de ce codage utilisant des actions
particulières appelées no-ops \cite{KMS96}.
Les règles du codage dans les espaces d'états produisent des propositions de la
forme $a(i)$ qui représentent le fait que l'action $a$ est appliquée à un niveau
$i$ du plan ssi $a(i)$ a la valeur de vérité vrai, et des propositions de la
forme $f(i)$ qui représentent le fait que le fluent $f$ est présent au niveau
$i$ ssi $f(i)$ a la valeur de vérité vrai. Le présence de $f$ au niveau $i$
signifie qu'il est présent après l'application successive de toutes les
actions associées à des propositions qui sont vraies, du niveau 1 jusqu'au
niveau $i$. Ce codage comporte cinq règles~:

} % FIN COLOR RED


%\begin{enumerate}
%\item \codrule
\paragraph*{[SAT-EFA.1 -- état initial et but]}
Les fluents de l'état initial sont vrais au niveau $0$ et ceux qui ne sont pas dans l'état initial sont faux au niveau $0$. Tous les fluents du but doivent être vrais au niveau $\length$.

  \codage{\noindent\left(\Bigforall{\f \in \I} \f_{0} \right) \land \left(\Bigforall{\f \in (\F \setminus \I)}\neg \f_{0} \right) \land \left(\Bigforall{\f \in \G}\f_{\length} \right)}
%\item \label{codage:ee2} \codrule
\paragraph*{[SAT-EFA.2 -- Préconditions et effets des actions]} %si une action appartient au plan, alors ses préconditions sont vérifiées et ses effets sont produits.
Si une action $\a$ est exécutée dans une transition du plan (au niveau $i$), alors chaque effet de $\a$ se produit dans l'état résultant (au niveau $i$) et chaque précondition de $\a$ est requise dans l'état précédent (au niveau $i-1$).
  \codage{\BigforallTwo{i=1}{\length} \Bigforall{\a \in \A} \left(
      \a_{i} \Rightarrow \left( \Bigforall{\f \in \Cond{\a}} \f_{i-1} \right) \land
      \left( \Bigforall{\f \in \Add{\a}} \f_{i} \right) \land \left( \Bigforall{\f
          \in \Del{\a}} \neg \f_{i} \right) \right)}
%\item \label{codage:ee4} \codrule
\paragraph*{[SAT-EFA.3 -- Frame-axiomes explicatifs]} %d'ajout]} %si un fluent devient vrai entre deux niveaux successifs du plan, alors une action au moins qui l'établit doit avoir été appliquée.  Il faut que le fluent puisse ne pas exister à un instant donné, donc ne pas appartenir à $I$ ou appartenir à $F_d$. Il lui faut aussi pouvoir être ajouté par une action donc appartenir à $F_a$.
Si la valeur d'un fluent change entre deux états consécutifs (du niveau $i-1$ au niveau $i$), alors une action qui produit cette modification est exécutée dans la transition du plan entre ces états (au niveau $i$).
  \codage{\BigforallTwo{i=1}{\length} \Bigforall{\f \in \F} \left( \neg \f_{i-1} \land \f_{i} \Rightarrow \Bigexists{\substack{\a \in \A \\ \f \in \Add{\a}}} \a_{i} \right)}
%\item \label{codage:ee3} \codrule
%\paragraph*{[SAT-EFA.3.2 -- Frame-axiomes explicatifs de retrait]} si un fluent devient faux entre deux niveaux successifs du plan, alors une action au moins qui le retire doit avoir été appliquée. Il faut que le fluent existe à un instant donné, il doit donc appartenir à $I$ ou à $F_a$. Il lui faut aussi pouvoir être retiré par une action, il doit donc aussi appartenir à $F_d$.
 \codage{\BigforallTwo{i=1}{\length} \Bigforall{\f \in \F} \left( \f_{i-1} \land \neg \f_{i} \Rightarrow \Bigexists{\substack{\a \in \A \\ \f \in \Del{\a}}} \a_{i} \right)}
%\item \label{codage:ee5} \codrule
\paragraph*{[SAT-EFA.4 -- Prévention des interactions négatives]}%Interférences]} %deux actions non indépendantes ne peuvent pas être exécutées au même niveau. 
Les effets contradictoires sont déjà pris en compte par la règle SAT-EFA.2.
Cette règle doit donc seulement empêcher les interactions entre les préconditions et les retraits des actions. Si une action supprime un fluent nécessaire à une autre action, ces deux actions ne peuvent pas être exécutées dans une même transition du plan.

%\begin{scriptsize}
\[ \BigforallTwo{i=0}{\depth} \Bigforall{\a \in \A} \Bigforall{\f\in \Cond{\a}} \Bigforall{\b \in \A \\ \a \neq \b \\ \f \in \Del{\b}}\left(\neg \a_{i} \vee \neg \b_{i}\right) \]
%\end{scriptsize}  
%  \codage{\biget{i \in [1,k]} \biget{\{a_m, a_n\} \subseteq A \suchthat m < n \land (a_m\|_e a_n) \land \lnot (a_m\|_i a_n)} \left[ \neg a_m(i) \lor \neg a_n(i) \right]}
%\end{enumerate}

