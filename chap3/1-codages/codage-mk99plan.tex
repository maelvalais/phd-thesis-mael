%
% codage SAT-MK99-ESPACES-DE-PLAN
%
\subsection{Codages SAT de référence dans les espaces de plans}

\fred{Présenter les codages MK99 espaces de plans}

\fred{Attention, ici copier/coller ouvrage SAT 2008 Maris, Régnier, Vidal}
{\color{red}
Ces codages ne traduisent plus seulement les transitions qui s'opèrent entre
niveaux successifs du plan, mais expriment maintenant les relations de causalité
entre les actions qui le constituent. Dans les espaces d'états, l'application
des actions est envisagée séquentiellement~: une action ne peut apparaître à un
niveau du plan que lorsque ses préconditions sont satisfaites au niveau
précédent.  Dans les espaces de plans, une action apparaît à un niveau du plan
parce qu'une action d'un niveau antérieur (pas forcément le niveau précédent)
établit ses préconditions et qu'une action qui suit requiert un de ses effets.
L'existence de différents codages découle de la traduction des différentes
stratégies de recherche utilisées dans les espaces de plans. Ces codages restent
moins performants, en termes de compacité et de temps de résolution, que le
meilleur codage dans les espaces d'états (celui avec frame-axiomes explicatifs)
\cite{MK99}.

Dans tous ces codages, les ensembles d'actions indépendants qui peuvent faire
partie d'un plan sont associés à des symboles d'étapes. Un ordre sur ces étapes
détermine un ordre total pour l'exécution des ensembles d'actions indépendants.
Deux étapes particulières sont créées~: l'étape $p_0$ qui représente l'état
initial du problème, et l'étape $p_{k+1}$ qui représente le but du problème. Ces
deux étapes ne contiendront aucune action. Elles sont équivalentes aux actions
fictives $Start$ et $End$ qui servent à initialiser un plan partiel et sont généralement
utilisées en planification dans les espaces de plans partiels. Pour prendre en
compte ces deux étapes, l'ensemble $F_a$ des fluents présents dans les ajouts
des actions est augmenté des fluents de l'état initial $I$. De même, l'ensemble
$F_p$ des fluents présents dans les préconditions des actions est augmenté des
fluents du but $B$. On peut produire des codages produisant moins de clauses en
considérant les actions $Start$ et $End$ comme des cas particuliers
\cite{Vid01}, mais pour des raisons de compacité dans l'écriture des codages
nous les présentons ici de manière intégrée aux autres actions.

Les règles des codages dans les espaces de plans produisent les propositions
suivantes, pour une action $a$, un fluent $f$, et deux étapes $p$ et $q$~:

\begin{itemize}
\item $(a \in p)$ est vraie ssi l'action $a$ appartient à l'étape $p$.
\item $\mathrm{Adds}(p, f)$, est vraie ssi l'étape $p$ contient une action qui ajoute le
  fluent $f$.
\item $\mathrm{Needs}(p, f)$ est vraie ssi l'étape $p$ contient une action qui a $f$ en
  précondition.
\item $\mathrm{Dels}(p, f)$ est vraie ssi l'étape $p$ contient une action qui retire le
  fluent $f$.
\item $p \causal{f} q$ représente un lien causal et est vraie ssi l'étape
  $p$ produit le fluent $f$ qui est une précondition de l'étape $q$.
\item $p \prec q$ est vraie ssi l'étape $p$ précède l'étape $q$~; toutes les actions
  associées à $p$ doivent donc être exécutées avant celles associées à $q$.
  Ces propositions traduisent un ordre partiel sur les étapes.
\end{itemize}


\subsubsection{Partie commune des codages dans les espaces de plans}

Elle permet d'établir une correspondance entre toutes les actions disponibles et
celles qui font partie des étapes des plans solutions. Elle est constituée
de cinq règles~:

\begin{enumerate}
\item \codrule{état initial et but~: } l'étape $p_0$ produit l'état
  initial du problème, et l'étape $p_{k+1}$ requiert les buts.
  \codage{\left[ \biget{f \in I}\mathrm{Adds}(p_0,f)\right] \land \left[ \biget{f \in (F\setminus I)}\lnot\mathrm{Adds}(p_0,f)\right] \land \left[ \biget{f \in B} \mathrm{Needs}(p_{k+1},f)\right]\label{codage:epcom1}}
\item \codrule{Correspondance action/étape~:} si une étape ajoute, retire, ou a
  pour précondition un fluent, alors cette étape peut correspondre à n'importe
  quelle action qui a le même comportement vis-à-vis de ce fluent. Inversement,
  si une action appartient à une étape, ses préconditions, ajouts et retraits
  sont vrais.

  \codage{\biget{i \in [1,k]} \biget{f \in F_a} \left[ \mathrm{Adds}(p_i,f) \Leftrightarrow \bigou{a \in A | f \in \mathrm{Add}(a)} (a \in p_i) \right]\label{codage:epcom2}}
  \codage{\biget{i \in [1,k]} \biget{f \in F_d} \left[ \mathrm{Dels}(p_i,f) \Leftrightarrow \bigou{a \in A | f \in \mathrm{Del}(a)} (a \in p_i) \right]\label{codage:epcom3}}
  \codage{\biget{i \in [1,k]} \biget{f \in F_p} \left[ \mathrm{Needs}(p_i,f) \Leftrightarrow \bigou{a \in A | f \in \mathrm{Prec}(a)} (a \in p_i) \right]\label{codage:epcom4}}
\item \codrule{Exclusions mutuelles~: } deux actions qui ne sont pas indépendantes ne peuvent appartenir à une même étape.
  \codage{\biget{i \in [1,k]} \biget{ \{a_m, a_n\} \subseteq A \suchthat m < n \land \lnot(a_m\|a_n)} \left[\neg (a_m \in p_i) \lor \neg(a_n \in p_i) \right]\label{codage:epcom5}}
\end{enumerate}

\subsubsection{Codage par liens causaux, protection d'intervalles et ordre partiel}

La production des préconditions se fait par le codage direct des liens causaux.
La proposition $p \causal{f} q$ traduit le fait que l'étape $p$ produit le fluent
$f$, précondition de l'étape $q$. La protection de $f$ dans l'intervalle compris
entre les deux étapes se fait par promotion (l'étape menaçant $f$ est placée
avant $p$), ou par rétrogradation (l'étape menaçant $f$ est placée après $q$).
L'ordre d'exécution des étapes  est donné par une relation de précédence de
la forme $p \prec q$. Les quatre règles suivantes sont ajoutées à la partie commune~:

\begin{enumerate}
\item \codrule{Production des préconditions~:} chaque fluent est supporté par un lien
  causal entre l'étape qui le produit et l'étape
  qui l'utilise en tant que précondition.
  \codage{\biget{i \in [1,k+1]} \biget{f\in F_p} \left[ \mathrm{Needs}(p_i, f) \Rightarrow \ifthen{f \in F_a}{\bigou{j \in [0, k] \suchthat j \neq i} p_j \causal{f} p_i}{\bot} \right]}
\item \codrule{Liens causaux~:} un lien causal supportant un fluent $f$ entre
  deux étapes $p_i$ et $p_j$ implique que $p_i$ ajoute $f$, que $p_j$ a pour
  précondition $f$, et que $p_i$ précède $p_j$.
  \codage{\biget{i \in [0,k]} \biget{j \in [1,k+1]\suchthat i\neq j} \biget{f\in (F_p \cap F_a)} \left[ p_i \causal{f} p_j \Rightarrow\left( \mathrm{Adds}(p_i, f) \land \mathrm{Needs}(p_j,f)\land p_i \prec p_j  \right) \right]}
\item \codrule{Protection d'intervalles~:} si un lien causal supporte le
  fluent $f$ entre une étape $p_i$ et une étape $p_j$, et si une étape $p_q$
  retire $f$, alors $p_q$ doit précéder $p_i$ (promotion) ou $p_j$ doit précéder
  $p_q$ (rétrogradation).
  \codage{\biget{i \in [0,k]} \biget{j \in [1,k+1]\suchthat i\neq j} \biget{q \in [1,k]\suchthat q\neq i \land q\neq j} \biget{f\in (F_p \cap F_a \cap F_d)} 
  \left[ \begin{array}{l}p_i \causal{f} p_j \land \mathrm{Dels}(p_q, f) \Rightarrow\\ \left( p_q \prec p_i \lor p_j \prec p_q \right) \end{array} \right]}
\item \codrule{Propriétés de la relation de précédence~:} transitive et
  antisymétrique. L'irréflexivité est codée dans les autres règles par le biais
  des indices des étapes.
  \codage{\biget{i \in [0,k+1]} \biget{j \in [0,k+1]\suchthat i\neq j} \biget{q \in [0,k+1]\suchthat q\neq i \land q\neq j} \left( \left( p_i \prec p_j \land p_j \prec p_q \right) \Rightarrow p_i \prec p_q \right)}
  \codage{\biget{i \in [0,k]} \biget{j \in [i+1,k+1]\suchthat i\neq j} \lnot \left( p_i \prec p_j \right) \lor \lnot \left( p_j \prec p_i \right)}
\end{enumerate}

\subsubsection{Codage par liens causaux, protection d'intervalles et étapes contiguës}

Ce codage simplifie le précédent dans lequel plusieurs modèles correspondent au
même plan solution, à la numérotation des étapes près. Les étapes doivent
maintenant suivre un ordre prédéfini d'indice croissant, ce qui rend inutile la
relation de précédence. La protection d'intervalles ne nécessite plus la
promotion ou la rétrogradation avec un ordre partiel sur les étapes, mais se
réalise en interdisant qu'une étape comprise dans un intervalle défini par un
lien causal ne le menace. Les deux règles suivantes sont ajoutées à la partie
commune~:

\begin{enumerate}
\item \codrule{Production des préconditions~:} chaque fluent est supporté par un
  lien causal entre l'étape qui le produit et l'étape qui l'utilise en tant que
  précondition.
  \codage{\biget{i \in [1,k]} \biget{f\in F_p} \left[ \mathrm{Needs}(p_i, f) \Rightarrow \ifthen{f \in F_a}{\bigou{j \in [1, i-1]} p_j \causal{f} p_i}{\bot}\right]}
\item \codrule{Liens causaux et protection d'intervalles~:} la présence d'un lien
  causal supportant un fluent $f$ entre deux étapes $p_i$ et $p_j$
  implique que $p_i$ ajoute $f$ et que $p_j$ a pour précondition
  $f$. De plus, une étape qui retire $f$ ne peut s'insérer entre $p_i$ et
  $p_j$.
  \codage{\biget{i \in [1,k-1]} \biget{j \in [i+1,k]} \biget{f\in (F_p \cap F_a)} \left[ \begin{array}{l} p_i \causal{f} p_j \Rightarrow \mathrm{Adds}(p_i, f) \land \mathrm{Needs}(p_j,f) \land \\ \ifthen{f \in F_d}{\biget{q\in [i+1,j-1]} \lnot \mathrm{Dels}(p_q, f)}{\top} \end{array}\right]}
\end{enumerate}

\subsubsection{Codage du \enquote{chevalier blanc}}

Ce codage est actuellement le plus compact des codages dans les espaces de plans
en nombre de variables et de clauses produites~; il est également le plus
performant par son temps de résolution. Il exprime directement les liens
causaux~: si une étape requiert un fluent, c'est qu'une étape précédente doit
l'avoir créé. On n'utilise ainsi que les variables déjà présentes dans la partie
commune. La protection d'intervalles se réalise en codant la technique du
\enquote{chevalier blanc} introduite par le planificateur \textsc{Tweak}
\cite{Cha87}. Les deux règles suivantes sont ajoutées à la partie commune~:


\begin{enumerate}
\item \codrule{Production des préconditions~:} la précondition d'une étape d'un niveau
  $i$ doit être ajoutée par une étape précédente.
  \codage{\biget{i \in [1,k+1]} \biget{f\in F_p} \left[ \mathrm{Needs}(p_i, f) \Rightarrow \ifthen{f \in F_a}{\bigou{j \in [0, i-1]} \mathrm{Adds}(p_j, f)}{\bot} \right]}

\item \codrule{Chevalier blanc~:} si une étape a pour précondition le fluent $f$
  au niveau $i$, et qu'une autre le retire au niveau $j$ avant que la première
  puisse l'utiliser, alors il doit y avoir une troisième étape qui rétablit $f$
  à un niveau $q$ tel que $j < q < i$.
  \codage{\biget{i \in [3,k+1]} \biget{j \in [1,i-2]} \biget{f \in (F_p \cap F_d)} \left[ \begin{array}{l}\mathrm{Needs}(p_i, f) \land \mathrm{Dels}(p_j, f) \Rightarrow\\ 
        \ifthen{f \in F_a}{\bigou{q \in [j+1,i-1]} \mathrm{Adds}(p_q,f)}{\bot}\end{array} \right]}
\end{enumerate}


}