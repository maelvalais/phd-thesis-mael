%
% codage SAT-MK99-ESPACES-DE-PLAN
%
\subsection{Codages SAT de référence dans les espaces de plans}

Les codages dans les espaces de plans ne traduisent pas uniquement les transitions entre niveaux successifs du plan. Ils expriment également les relations de causalité entre les actions qui le constituent. Dans les espaces d'états, l'application des actions est envisagée séquentiellement~: une action ne peut apparaître à un niveau du plan que lorsque ses préconditions sont satisfaites au niveau précédent.  Dans les espaces de plans, une action est présente à un niveau du plan parce qu'une action d'un niveau inférieur (pas forcément le précédent) établit ses préconditions et qu'une action d'un niveau supérieur nécessite l'un de ses effets.
L'existence de différents codages découle de la traduction des différentes stratégies de recherche utilisées dans les espaces de plans. %Ces codages restent moins performants, en termes de compacité et de temps de résolution, que le meilleur codage dans les espaces d'états (celui avec frame-axiomes explicatifs)
%\cite{MK99}.

Nous allons maintenant décrire la partie commune SAT-CL des codages de référence dans les espaces de plans avant de détailler les règles de codage complémentaires des trois codages proposés par ~\cite{MK99} :
\begin{itemize}
    \item SAT-CLPO : codage par liens causaux, protection d'intervalles et ordre partiel ;
    \item SAT-CLCS : codage par lien causaux, protection d'intervalles et étapes contiguës ;
    \item SAT-CLWK : codage du \enquote{chevalier blanc}.
\end{itemize}

Dans tous ces codages, des ensembles d'actions indépendants sont associés à des étapes d'un plan, notées $\step{i}$. Un ordre sur ces étapes permet de déterminer un ordre total pour l'exécution des ensembles d'actions indépendants.
Deux étapes particulières qui ne contiennent aucune action sont créées~: l'étape $\step{0}$ qui représente l'état initial $\I$, et l'étape $\step{k+1}$ qui représente le but $\G$.
%Ces deux étapes ne contiendront aucune action. Elles sont équivalentes aux actions fictives $Start$ et $End$ qui servent à initialiser un plan partiel et sont généralement utilisées en planification dans les espaces de plans partiels. Pour prendre en compte ces deux étapes, l'ensemble $F_a$ des fluents présents dans les ajouts des actions est augmenté des fluents de l'état initial $I$. De même, l'ensemble $F_p$ des fluents présents dans les préconditions des actions est augmenté des fluents du but $B$. On peut produire des codages produisant moins de clauses en considérant les actions $Start$ et $End$ comme des cas particuliers \cite{Vid01}, mais pour des raisons de compacité dans l'écriture des codages nous les présentons ici de manière intégrée aux autres actions.


\fred{L'ensemble de variables propositionnelles $\X$ utilisées pour ce codage est ainsi donné par $\X = \X_{\F} \cup \X_{\A} \cup \X_{\mathcal{N}}$, où $X_{\F}$ et $X_{\A}$ sont les ensembles de variables de fluents et d'actions identiques au codage SAT-EFA, et :
\begin{itemize}
    \item l'ensemble des variables de No-ops est défini par $\displaystyle\X_{\mathcal{N}}=\bigcup_{i=1}^{\length}\{\noop{f}{i}\mid \f\in\F\}$.
\end{itemize}}

Les règles des codages dans les espaces de plans produisent les propositions suivantes, pour une action $\a$, un fluent $\f$, et deux étapes $\step{i}$ et $\step{j}$~:

\begin{itemize}
\item $(\a \in \step{i})$ est vraie ssi l'action $\a$ appartient à l'étape $\step{i}$.
\item $\Adds{i}{\f}$, est vraie ssi l'étape $\step{i}$ contient une action qui ajoute le
  fluent $\f$.
\item $\Needs{i}{\f}$ est vraie ssi l'étape $\step{i}$ contient une action qui a $f$ en
  précondition.
\item $\Dels{i}{\f}$ est vraie ssi l'étape $\step{i}$ contient une action qui retire le
  fluent $\f$.
\item $\Linksteps{i}{\f}{j}$ représente un lien causal et est vraie ssi l'étape
  $\step{i}$ produit le fluent $\f$ qui est une précondition de l'étape $\step{j}$.
\item $\step{i} \prec \step{j}$ est vraie ssi l'étape $\step{i}$ précède l'étape $\step{j}$~; toutes les actions
  associées à $\step{i}$ doivent donc être exécutées avant celles associées à $\step{j}$.
  Ces propositions traduisent un ordre partiel sur les étapes.
\end{itemize}


{\color{red}

\subsubsection{Partie commune des codages dans les espaces de plans}

Elle permet d'établir une correspondance entre toutes les actions disponibles et
celles qui font partie des étapes des plans solutions. Elle est constituée
de cinq règles~:


%\begin{enumerate}
%\item 
%\tagrule{SAT/PS-MK99.1}\codrule
\paragraph*{[SAT-CL.1 -- état initial et but]} l'étape $\step{0}$ produit l'état
  initial du problème, et l'étape $\step{k+1}$ requiert les buts.
  \codage{\left( \Bigforall{\f \in \I}\Adds{0}{\f}\right) \land \left( \Bigforall{\f \in (\F\setminus \I)}\lnot\Adds{0}{\f}\right) \land \left( \Bigforall{\f \in \G} \Needs{k+1}{\f}\right)\label{codage:epcom1}}
%\item 
%\tagrule{SAT/PS-MK99.2}\codrule
\paragraph*{[SAT-CL.2 -- Correspondance action/étape]} si une étape ajoute, retire, ou a
  pour précondition un fluent, alors cette étape peut correspondre à n'importe
  quelle action qui a le même comportement vis-à-vis de ce fluent. Inversement,
  si une action appartient à une étape, ses préconditions, ajouts et retraits
  sont vrais.

  \codage{\BigforallTwo{i=1}{k} \Bigforall{\f \in \F} \begin{pmatrix}
  \left( \Adds{i}{\f} \Leftrightarrow \Bigexists{\substack{\a \in \A \\\f \in \Add{\a}}} \instep{\a}{i} \right)
  \wedge \left( \Dels{i}{\f} \Leftrightarrow \Bigexists{\substack{\a \in \A \\\f \in \Del{\a}}} \instep{\a}{i} \right)\\
  \wedge \left( \Needs{i}{\f} \Leftrightarrow \Bigexists{\substack{\a \in \A \\\f \in \Cond{\a}}} \instep{\a}{i} \right)
  \end{pmatrix}}

%  \codage{\BigforallTwo{i=1}{k} \Bigforall{\f \in \F} \left( \Adds{i}{\f} \Leftrightarrow \Bigexists{\substack{\a \in \A \\\f \in \Add{\a}}} \instep{\a}{i} \right)\label{codage:epcom2}}
%  \codage{\BigforallTwo{i=1}{k} \Bigforall{\f \in \F} \left( \Dels{i}{\f} \Leftrightarrow \Bigexists{\substack{\a \in \A \\\f \in \Del{\a}}} \instep{\a}{i} \right)\label{codage:epcom3}}
%  \codage{\BigforallTwo{i=1}{k} \Bigforall{\f \in \F} \left( \Needs{i}{\f} \Leftrightarrow \Bigexists{\substack{\a \in \A \\\f \in \Cond{\a}}} \instep{\a}{i} \right)\label{codage:epcom4}}
%\item 
%\tagrule{SAT/PS-MK99.3}\codrule
%\paragraph*{[SAT-MK99.3 -- Exclusions mutuelles]} deux actions qui ne sont pas indépendantes ne peuvent appartenir à une même étape.
%  \codage{\biget{i \in [1,k]} \biget{ \{a_m, a_n\} \subseteq A \suchthat m < n \land \lnot(a_m\|a_n)} \left[\neg (a_m \in p_i) \lor \neg(a_n \in p_i) \right]\label{codage:epcom5}}
%\end{enumerate}

\paragraph*{[SAT-CL.3 -- Prévention des interactions négatives]}
Si une action supprime un fluent qui est nécessaire ou est ajouté par une autre action, alors ces deux actions ne peuvent pas être exécutées toutes les deux dans une même étape.

\begin{small}
\[
\BigforallTwo{i=1}{k}\Bigforall{\substack{\a\in \A}}\Bigforall{\substack{\f\in \left(\Add{\a}\cup\Cond{\a}\right)}}\Bigforall{\substack{\b\in \A\\\a \neq \b \wedge \f \in \Del{\b}}}\left(\neg \instep{\a}{i} \vee \neg \instep{\b}{i}\right)
\]
\end{small}

\subsubsection{Codage par liens causaux, protection d'intervalles et ordre partiel}

La production des préconditions se fait par le codage direct des liens causaux.
La proposition $\Linksteps{i}{\f}{j}$ traduit le fait que l'étape $\step{i}$ produit le fluent
$\f$, précondition de l'étape $\step{j}$. La protection de $\f$ dans l'intervalle compris
entre les deux étapes se fait par promotion (l'étape menaçant $\f$ est placée
avant $\step{i}$), ou par rétrogradation (l'étape menaçant $\f$ est placée après $\step{j}$).
L'ordre d'exécution des étapes  est donné par une relation de précédence de
la forme $\step{i} \prec \step{j}$. Les quatre règles suivantes sont ajoutées à la partie commune~:

%\begin{enumerate}
%\item \codrule
\paragraph*{[SAT-CLPO.4 -- Production des préconditions]} chaque fluent est supporté par un lien
  causal entre l'étape qui le produit et l'étape
  qui l'utilise en tant que précondition.
  \codage{\BigforallTwo{i=1}{k+1} \Bigforall{\f\in \F} \left( \Needs{i}{\f} \Rightarrow %\ifthen{\f \in \F_a}{
  \BigexistsTwo{\substack{j=0\\j\neq i}}{k} \Linksteps{j}{\f}{i}%}{\bot}
  \right)}
%\item \codrule
\paragraph*{[SAT-CLPO.5 -- Liens causaux]} un lien causal supportant un fluent $\f$ entre
  deux étapes $\step{i}$ et $\step{j}$ implique que $\step{i}$ ajoute $\f$, que $\step{j}$ a pour
  précondition $\f$, et que $\step{i}$ précède $\step{j}$.
  \codage{\BigforallTwo{i=0}{k} \BigforallTwo{\substack{j=1\\j\neq i}}{k+1} \Bigforall{\f\in \F} \left( \Linksteps{i}{\f}{j} \Rightarrow\left( \Adds{i}{\f} \land \Adds{j}{\f}\land \step{i} \prec \step{j}  \right) \right)}
%\item \codrule
\paragraph*{[SAT-CLPO.6 -- Protection d'intervalles]} si un lien causal supporte le
  fluent $\f$ entre une étape $\step{i}$ et une étape $\step{j}$, et si une étape $\step{q}$
  retire $\f$, alors $\step{q}$ doit précéder $\step{i}$ (promotion) ou $\step{j}$ doit précéder
  $\step{q}$ (rétrogradation).
  \codage{\BigforallTwo{i=0}{k} \BigforallTwo{\substack{j=1\\j\neq i}}{k+1} \BigforallTwo{\substack{q=1\\q\neq i \land q\neq j}}{k} \Bigforall{\f\in \F} 
  \left( %\begin{array}{l}
  \Linksteps{i}{\f}{j} \land \Dels{q}{\f} \Rightarrow%\\
  \left( \step{q} \prec \step{i} \lor \step{j} \prec \step{q} \right) %\end{array}
  \right)}
%\item \codrule
\paragraph*{[SAT-CLPO.7 -- Propriétés de la relation de précédence]} transitive et
  antisymétrique. L'irréflexivité est codée dans les autres règles par le biais
  des indices des étapes.
  \codage{\BigforallTwo{i=0}{k+1} \BigforallTwo{\substack{j=0\\j\neq i}}{k+1} \BigforallTwo{\substack{q=0\\q\neq i \land q\neq j}}{k+1} \left( \left( \step{i} \prec \step{j} \land \step{j} \prec \step{q} \right) \Rightarrow \step{i} \prec \step{q} \right)}
  \codage{\BigforallTwo{i=0}{k} \BigforallTwo{\substack{j=i+1\\j\neq i}}{ k+1} \lnot \left( \step{i} \prec \step{j} \right) \lor \lnot \left( \step{j} \prec \step{i} \right)}
%\end{enumerate}

\subsubsection{Codage par liens causaux, protection d'intervalles et étapes contiguës}

Ce codage simplifie le précédent dans lequel plusieurs modèles correspondent au
même plan solution, à la numérotation des étapes près. Les étapes doivent
maintenant suivre un ordre prédéfini d'indice croissant, ce qui rend inutile la
relation de précédence. La protection d'intervalles ne nécessite plus la
promotion ou la rétrogradation avec un ordre partiel sur les étapes, mais se
réalise en interdisant qu'une étape comprise dans un intervalle défini par un
lien causal ne le menace. Les deux règles suivantes sont ajoutées à la partie
commune~:

%\begin{enumerate}
%\item \codrule
\paragraph*{[SAT-CLCS.4 -- Production des préconditions]} chaque fluent est supporté par un
  lien causal entre l'étape qui le produit et l'étape qui l'utilise en tant que
  précondition.
  \codage{\BigforallTwo{i=1}{k} \Bigforall{\f\in \F} \left( \Needs{i}{\f} \Rightarrow %\ifthen{\f \in \F_a}{
  \BigexistsTwo{j=1}{i-1} \Linksteps{j}{\f}{i}%}{\bot}
  \right)}
%\item \codrule
\paragraph*{[SAT-CLCS.5 -- Liens causaux et protection d'intervalles]} la présence d'un lien
  causal supportant un fluent $\f$ entre deux étapes $\step{i}$ et $\step{j}$
  implique que $\step{i}$ ajoute $\f$ et que $\step{j}$ a pour précondition
  $\f$. De plus, une étape qui retire $\f$ ne peut s'insérer entre $\step{i}$ et $\step{j}$.
%  \codage{
\[
\BigforallTwo{i=1}{k-1} \BigforallTwo{j=i+1}{k} \Bigforall{\f\in \F} \left( %\begin{array}{l} 
  \Linksteps{i}{\f}{j} \Rightarrow \Adds{i}{\f} \land \Needs{j}{\f} \land %\\ %\ifthen{f \in F_d}{
  \BigforallTwo{q=i+1}{j-1} \lnot \Dels{q}{\f}%}{\top}
  %\end{array}
  \right)
\]
  %}
%\end{enumerate}

\subsubsection{Codage du \enquote{chevalier blanc}}

Ce codage est actuellement le plus compact des codages dans les espaces de plans
en nombre de variables et de clauses produites~; il est également le plus
performant par son temps de résolution. Il exprime directement les liens
causaux~: si une étape requiert un fluent, c'est qu'une étape précédente doit
l'avoir créé. On n'utilise ainsi que les variables déjà présentes dans la partie
commune. La protection d'intervalles se réalise en codant la technique du
\enquote{chevalier blanc} introduite par le planificateur \textsc{Tweak}
\cite{Cha87}. Les deux règles suivantes sont ajoutées à la partie commune~:


%\begin{enumerate}
%\item \codrule
%\tagrule{SAT/WK.1}\codrule
\paragraph*{[SAT-CLWK.4 -- Production des préconditions]} La précondition d'une étape d'un niveau
  $i$ doit être ajoutée par une étape précédente.
  \codage{\BigforallTwo{i=1}{k+1} \Bigforall{\f\in \F} \left( \Needs{i}{\f} \Rightarrow %\ifthen{f \in F_a}{
  \BigexistsTwo{j=0}{i-1} \Adds{j}{\f}%}{\bot}
  \right)}

%\item \codrule
%\tagrule{SAT/WK.2}\codrule
\paragraph*{[SAT-CLWK.5 -- Chevalier blanc]} Si une étape a pour précondition le fluent $f$
  au niveau $i$, et qu'une autre le retire au niveau $j$ avant que la première
  puisse l'utiliser, alors il doit y avoir une troisième étape qui rétablit $f$
  à un niveau $q$ tel que $j < q < i$.
  \codage{\BigforallTwo{i=3}{k+1} \BigforallTwo{j=1}{i-2} \Bigforall{\f \in \F} \left( %\begin{array}{l}
  \Needs{i}{\f} \land \Dels{j}{\f} \Rightarrow%\\ 
        %\ifthen{f \in F_a}{
        \BigexistsTwo{q=j+1}{i-1} \Adds{q}{\f}%}{\bot}
        %\end{array}
        \right)}
%\end{enumerate}

} % FIN COLOR RED
