%
% SMT TLP-GP
%

Plusieurs codages SMT pour la planification temporelle ont déjà été proposés. La plupart d'entre eux sont basés sur une représentation discrète du temps \cite{DBLP:journals/ai/ShinD05,DBLP:conf/aaai/Rintanen15}. Malheureusement, cette représentation ne permet pas de résoudre tous les problèmes de planification temporelle~\cite{DBLP:journals/ci/CooperMR13}, et c'est pour cette raison que nous ne présentons ici que le codage de référence de TLP-GP \cite{DBLP:conf/ictai/MarisR08} basé sur une représentation continue du temps et l'utilisation d'atomes de QF-RDL (Quantifier Free Rational Difference Logic). Le codage que nous détaillons, et que nous noterons SMT-LINK, est en réalité une réécriture plus générique du codage original qui nous permet d'éviter de faire référence aux différents arcs d'un graphe de planification \cite{BF95,BF97}.

Ici, contrairement aux codages de problèmes de planification classique, les actions ne sont plus instantanées et des contraintes sur les instants $\tau(e)$ auxquels se produisent des événements $e\in\events{\A}$ doivent explicitement être ajoutées.
Considérons donc maintenant un problème de planification temporelle positif $\langle \I, \A, \G \rangle$.
L'ensemble de variables propositionnelles $\X$ utilisées pour le codage SMT-LINK est donné par $\X = \X_{\A} \cup \X_{\mathit{Link}}$, où :
\begin{itemize}
    \item $\displaystyle\X_{\A}=\{\mathit{Init}, \mathit{Goal}\} \cup \bigcup_{i=1}^{\length}\{\a_{i}\mid \a\in\A\}$ est l'ensemble des variables d'actions ;
    \item $\displaystyle\X_{\mathit{Link}}=\bigcup_{i=1}^{\length}\bigcup_{j=1}^{\length}\{\Link{\b_{i}}{\f}{\a_{j}}\mid (\b,\a)\in \A^{2} \wedge \f\in\add{\b}\cap\cond{\a}\}$ est l'ensemble des variables de liens causaux.
\end{itemize}

Dans ce codage, nous utilisons une action factice $\mathit{Init}$ qui produit les fluents de l'état initial et une action factice $\mathit{Goal}$ qui nécessite les fluents du but. La proposition $\Link{\b_{i}}{\f}{\a_{j}}$ traduit le fait que l'action $\b_{i}$ produit le fluent
$\f$, condition de l'action $\a_{j}$.

L'ensemble des variables rationnelles $T$ utilisées pour les atomes de QF-RDL de ce codage est donné par $\displaystyle T=\{\tau_{\mathit{Init}}, \tau_{\mathit{Goal}}\} \cup T_{\mathit{Cond_{s}}} \cup T_{\mathit{Cond_{e}}} \cup T_{\mathit{Add}} \cup T_{\mathit{Del}}$, où :

\begin{itemize}
\item $\tau_{\mathit{Init}}$ est la variable d'instant de début du plan auquel l'état initial doit être vérifié et $\tau_{\mathit{Goal}}$ est la variable d'instant de fin du plan auquel le but doit être obtenu ;
\item $\displaystyle T_{\mathit{Cond_{s}}}=\bigcup_{i=1}^{\length}\{\tau(\f\mid\rightarrow\a_{i})\mid \a\in\A\ \wedge \f\in\cond{\a}\}$ est l'ensemble des variables de début d'intervalles de conditions requises par les actions ;
\item $\displaystyle T_{\mathit{Cond_{e}}}=\bigcup_{i=1}^{\length}\{\tau(\f\rightarrow\mid\a_{i})\mid \a\in\A\ \wedge \f\in\cond{\a}\}$ est l'ensemble des variables de fin d'intervalles de conditions requises par les actions ;
\item $\displaystyle T_{\mathit{Add}}=\bigcup_{i=1}^{\length}\{\tau(\a_{i}\rightarrow\f)\mid \a\in\A\ \wedge \f\in\add{\a}\}$ est l'ensemble des variables d'instants d'ajouts de fluents par les actions ;
\item $\displaystyle T_{\mathit{Del}}=\bigcup_{i=1}^{\length}\{\tau(\a_{i}\rightarrow\neg\f)\mid \a\in\A\ \wedge \f\in\del{\a}\}$ est l'ensemble des variables d'instants de retraits de fluents par les actions.
\end{itemize}

\paragraph*{[SMT-LINK.1 -- Etat initial et But]}

Les nœuds d'actions factices Init (produisant l'état initial) et Goal (nécessitant le but) sont tous deux vrais.

\[
\mathit{Init} \wedge \mathit{Goal}
\]

\paragraph*{[SMT-LINK.2 -- Production des préconditions par liens causaux]}

Si une action $\a_{i}$ est active dans le plan à une étape $i$, alors pour chacune de ses préconditions $\f$, il existe au moins un lien causal (noté $\Link{\b_{j}}{\f}{\a_{i}}$) d'une action $\b_{j}$, qui produit cette précondition à l'étape $j$, vers $\a_{i}$.

\[ \BigforallTwo{i=1}{i=\length} \Bigforall{\a\in\A} \left( \a_{i} \Rightarrow \Bigforall{\f\in\Cond{\a}} \BigexistsTwo{j=1}{j=i}\Bigexists{\b\in\A\\\f\in\Add{\b}} \Link{\b_{j}}{\f}{\a_{i}} \right)
\]

\paragraph*{[SMT-LINK.3 -- Activation des actions et ordre partiel]}

S’il existe un lien causal entre une action $\b_{j}$ qui produit une précondition $\f$ pour une action $\a_{i}$, alors $\b_{j}$ et $\a_{i}$ sont actives dans le plan et l’instant où $\b_{j}$ produit certainement $\f$ est antérieur ou égal à l’instant où $\a_{j}$ commence à nécessiter $\f$.

\[
\begin{matrix}
\BigforallTwo{i=1}{i=\length} \BigforallTwo{j=1}{i=j} \Bigforall{(\a,\b)\in\A^{2}} \Bigforall{\f\in\left(\Cond{\a}\cap\Add{\b}\right)}\\ \left( \Link{\b_{j}}{\f}{\a_{i}} \Rightarrow \left( \b_{j} \wedge \a_{i} \wedge \tau(\b_{j}\rightarrow\f) \leq \tau(\f\mid\rightarrow\a_{i}) \right) \right)
\end{matrix}
\]

\paragraph*{[SMT-LINK.4 -- Protection des liens causaux]}

Si un lien causal assure la protection d’un fluent $\f$ et qu'une action qui le détruit est active dans le plan, alors l’intervalle temporel correspondant au lien causal et l’intervalle temporel correspondant à l’activation de $\neg \f$ (la destruction de $\f$) par l’action sont disjoints.

\[
\begin{matrix}
\BigforallTwo{i=1}{i=\length} \BigforallTwo{j=1}{i=j} \BigforallTwo{k=1}{k=\length} \Bigforall{(\a,\b)\in\A^{2}} \Bigforall{\f\in\left(\Cond{\a}\cap\Add{\b}\right)} \Bigforall{\otheraction\in\A\\\f\in\Del{\otheraction}}\\ \left( \Link{\b_{j}}{\f}{\a_{i}} \wedge \otheraction_{k} \Rightarrow \left( \begin{matrix} \tau(\otheraction_{k}\rightarrow\neg\f) < \tau(\b_{j}\rightarrow\f)\\ \vee \tau(\f\rightarrow\mid\a_{i}) < \tau(\otheraction_{k}\rightarrow\neg\f) \end{matrix} \right) \right)
\end{matrix}
\]

\paragraph*{[SMT-LINK.5 -- Prévention des interactions négatives]}

Si deux actions produisant respectivement une proposition $p$ et sa négation sont actives dans le plan, alors les intervalles temporels correspondants à l’activation de $p$ et à l’activation de $\neg p$ sont disjoints.

\[
\begin{matrix}
\BigforallTwo{i=1}{i=\length} \BigforallTwo{j=1}{j=\length} \Bigforall{(\a,\b)\in\A^{2}} \Bigforall{\f\in\left(\Add{\a}\cap\Del{\b}\right)}\\ \left( \a_{i} \wedge \b_{j} \Rightarrow \left( \begin{matrix} \tau(\a_{i}\rightarrow\neg\f) < \tau(\b_{j}\rightarrow\f)%\\ \vee \tau(\f\rightarrow\mid\a_{i}) < \tau(\otheraction_{k}\rightarrow\neg\f) 
\end{matrix} \right) \right)
\end{matrix}
\]

\paragraph*{[SMT-LINK.6 -- Bornes inférieure et supérieure]}

L’instant initial où les propositions de l’état initial sont vraies est antérieur à tous les instants de début des préconditions des actions du plan. L’instant final où les propositions du but sont vraies est postérieur à tous les instants de fin des effets des actions du plan.

\begin{small}
\[
\BigforallTwo{i=1}{\length}~\Bigforall{\substack{\a\in \A}}%\hfill\\
\left(\a_{i} \Rightarrow \begin{pmatrix}\Bigforall{\substack{\f\in \cond{\a}}} \begin{pmatrix}\left(\tau_{\mathit{Init}} \leq \tau(\f\mid\rightarrow\a_{i}) \right)%\hfill\\%
 \wedge \left(\tau_{\mathit{Goal}} \geq \tau(\f\rightarrow\mid\a_{i})\right) \end{pmatrix}\hfill\\
 \wedge \Bigforall{\substack{\f\in \add{\a}}} \begin{pmatrix}\left(\tau_{\mathit{Init}} \leq \tau(\a_{i}\rightarrow \f) \right)%\hfill\\%
 \wedge \left(\tau_{\mathit{Goal}} \geq \tau(\a_{i}\rightarrow \f)\right) \end{pmatrix}\hfill\\
 \wedge \Bigforall{\substack{\f\in \del{\a}}} \begin{pmatrix}\left(\tau_{\mathit{Init}} \leq \tau(\a_{i}\rightarrow \neg \f) \right)%\hfill\\%
 \wedge \left(\tau_{\mathit{Goal}} \geq \tau(\a_{i}\rightarrow \neg \f)\right) \end{pmatrix} \end{pmatrix}\right)
\]
\end{small}