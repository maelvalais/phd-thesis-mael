%
% SMT TLP-GP
%

Plusieurs codages SMT pour la planification temporelle ont déjà été proposés. La plupart d'entre eux sont basés sur une représentation discrète du temps \cite{DBLP:journals/ai/ShinD05,DBLP:conf/aaai/Rintanen15}. Malheureusement, cette représentation ne permet pas de résoudre tous les problèmes de planification temporelle, et c'est pour cette raison que nous ne présentons ici que le codage de référence de TLP-GP \cite{DBLP:conf/ictai/MarisR08} basé sur une représentation continue du temps et l'utilisation d'atomes de QF-RDL (Quantifier Free Rational Difference Logic). Le codage que nous détaillons est en réalité une réécriture plus générique du codage original qui nous permet d'éviter de faire référence aux différents arcs d'un graphe de planification \cite{BF95,BF97}.

\paragraph{Etat initial et But}

Les nœuds d'actions factices Init (produisant l'état initial) et Goal (nécessitant le but) sont tous deux vrais.

\paragraph{Production des préconditions par liens causaux}

Si une action $\a_{i}$ est active dans le plan à une étape $i$, alors pour chacune de ses préconditions $\f$, il existe au moins un lien causal (noté $\Link{\b_{j}}{\f}{\a_{i}}$) d'une action $\b_{j}$, qui produit cette précondition à l'étape $j$, vers $\a_{i}$.

\[ \BigforallTwo{i=1}{i=\length} \Bigforall{\a\in\A} \left( \a_{i} \Rightarrow \Bigforall{\f\in\Cond{\a}} \BigexistsTwo{j=1}{j=i}\Bigexists{\b\in\A\\\f\in\Add{\b}} \Link{\b_{j}}{\f}{\a_{i}} \right)
\]

\paragraph{Activation des actions et ordre partiel}

S’il existe un lien causal entre une action $\b_{j}$ qui produit une précondition $\f$ pour une action $\a_{i}$, alors $\b_{j}$ et $\a_{i}$ sont actives dans le plan et l’instant où $\b_{j}$ produit certainement $\f$ est antérieur ou égal à l’instant où $\a_{j}$ commence à nécessiter $\f$.

\[
\begin{matrix}
\BigforallTwo{i=1}{i=\length} \BigforallTwo{j=1}{i=j} \Bigforall{(\a,\b)\in\A^{2}} \Bigforall{\f\in\left(\Cond{\a}\cap\Add{\b}\right)}\\ \left( \Link{\b_{j}}{\f}{\a_{i}} \Rightarrow \left( \b_{j} \wedge \a_{i} \wedge \tau(\b_{j}\rightarrow\f) \leq \tau(\f\mid\rightarrow\a_{i}) \right) \right)
\end{matrix}
\]

\paragraph{Protection des liens causaux}

Si un lien causal assure la protection d’un fluent $\f$ et qu'une action qui le détruit est active dans le plan, alors l’intervalle temporel correspondant au lien causal et l’intervalle temporel correspondant à l’activation de $\neg \f$ (la destruction de $\f$) par l’action sont disjoints.

\[
\begin{matrix}
\BigforallTwo{i=1}{i=\length} \BigforallTwo{j=1}{i=j} \BigforallTwo{k=1}{k=\length} \Bigforall{(\a,\b)\in\A^{2}} \Bigforall{\f\in\left(\Cond{\a}\cap\Add{\b}\right)} \Bigforall{\otheraction\in\A\\\f\in\Del{\otheraction}}\\ \left( \Link{\b_{j}}{\f}{\a_{i}} \wedge \otheraction_{k} \Rightarrow \left( \begin{matrix} \tau(\otheraction_{k}\rightarrow\neg\f) < \tau(\b_{j}\rightarrow\f)\\ \vee \tau(\f\rightarrow\mid\a_{i}) < \tau(\otheraction_{k}\rightarrow\neg\f) \end{matrix} \right) \right)
\end{matrix}
\]

\paragraph{Prévention des interactions négatives}

Si deux actions produisant respectivement une proposition $p$ et sa négation sont actives dans le plan, alors les intervalles temporels correspondants à l’activation de $p$ et à l’activation de $\neg p$ sont disjoints.

\[
\begin{matrix}
\BigforallTwo{i=1}{i=\length} \BigforallTwo{j=1}{j=\length} \Bigforall{(\a,\b)\in\A^{2}} \Bigforall{\f\in\left(\Add{\a}\cap\Del{\b}\right)}\\ \left( \a_{i} \wedge \b_{j} \Rightarrow \left( \begin{matrix} \tau(\a_{i}\rightarrow\neg\f) < \tau(\b_{j}\rightarrow\f)%\\ \vee \tau(\f\rightarrow\mid\a_{i}) < \tau(\otheraction_{k}\rightarrow\neg\f) 
\end{matrix} \right) \right)
\end{matrix}
\]

\paragraph{Bornes inférieure et supérieure}

L’instant initial où les propositions de l’état initial sont vraies est antérieur à tous les instants de début des préconditions des actions du plan. L’instant final où les propositions du but sont vraies est postérieur à tous les instants de fin des effets des actions du plan.
\fred{à ajouter}
