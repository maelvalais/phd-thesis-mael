\fred{Intro à faire}

\section{Introduction générale au domaine}

\fred{La compilation de problèmes en logique bla bla résolution avec des solveurs bla bla}

\section{Le cadre de travail}

\subsection{Les solveurs pour les logiques classiques}

\subsubsection{SAT : satisfaction de formules booléennes}

\subsubsection{SMT : SAT Modulo Theories}
%\subsubsection{SMT : SAT Modulo Theories}

Certains probl\`emes combinatoires n\'ecessitent n\'eanmoins de traiter des calculs sur les nombre naturels ou r\'eels. Ceci peut \^etre fait en utilisant seulement la logique propositionnelle (par exemple, $2+3=5$ pourrait \^etre cod\'e par $add_{2,3,5}$), mais c'est tr\`es inconfortable \`a moins qu'il n'y ait que quelques additions \`a faire. Ne parlons m\^eme pas des op\'erations de multiplication ou plus complexes. L'id\'ee derri\`ere la gen\`ese de SMT a \'et\'e de combiner des solveurs SAT avec un solveur arithm\'etique dans le but d'am\'eliorer le traitement de la partie arithm\'etique du raisonnement. Dans de nombreux cas, ceci n'am\'eliorera pas seulement l'efficacit\'e du solveur, mais permettra aussi d'exprimer les contraintes arithm\'etiques des probl\`emes d'une mani\`ere radicalement plus compacte.

Pensez au jeu de Kamaji\footnote{\texttt{http://fr.wikipedia.org/wiki/Kamaji}}  o\`u le joueur doit grouper des nombres adjacents dans une grille de sorte que leur somme soit \'egale \`a un nombre fixe. R\'esoudre le jeu n\'ecessite essentiellement un raisonnement logique mais a aussi besoin d'un peu d'arithm\'etique (addition).


Si $x_{i,j}$ pour chaque case $(i,j)$ est un entier et $G(i,j,i,k)$ repr\'esente le fait que les cases $(i,j)$ \`a $(i,k)$ de la ligne $i$ forment un groupe, la contrainte de somme pourrait \^etre exprim\'ee par :
$$\sum_{m\in E}x_{i,m}=N$$
o\`u $N$ est le nombre fixe et $E$ est $\{j,j+1,\ldots,k\}$. La logique propositionnelle pure n'est certainement pas adapt\'ee pour de telles phrases !

\subsubsection{QBF : formules booléennes quantifiées}
Quantified Boolean Formula (QBF) est connu comme étant le problème de référence pour la classe de complexité PSPACE (\cite{Stockmeyer:1973:WPR:800125.804029}). C'est une extension de la logique propositionnelle permettant de quantifier sur les variables propositionnelles. 

Par exemple, $\forall p \exists q.p \leftrightarrow q$ se lit : pour toute valeur de vérité de $p$, il existe une valeur de vérité de $q$ tel que $p \leftrightarrow q$ est vrai. Cette formule est vraie (il suffit de choisir la même valeur pour $q$ que pour $p$). Alors que  $\exists p \forall q.p \lor q$ ne l'est pas. Ainsi, une formule booléenne quantifiée est toujours SOIT vraie SOIT fausse. 

De fait, à toute formule QBF peut être associée une formule propositionnelle sans variables car par définition : 
$\forall p.\Phi$ est vraie ssi $\Phi_{[p:=\top]} \wedge \Phi_{[p:=\bot]}$ l'est, et $\exists p.\Phi$ est vraie ssi $\Phi_{[p:=\top]} \vee \Phi_{[p:=\bot]}$. 

La formule QBF peut être exponentiellement plus compacte que la formule propositionnelle correspondante. 

Par exemple à la formule $\forall p \exists q.p \leftrightarrow q$ correspond la formule propositionnelle \\
$\Big ( (\top \leftrightarrow \top)\vee (\top \leftrightarrow \bot) \Big ) \wedge \Big ( (\bot \leftrightarrow \top)\vee (\bot \leftrightarrow \bot) \Big )$

\subsection{La planification par compilation automatique}
%%%%%%%%%%%%%%%%%%%%%%%%%%%%%%%
% PLANIFICATION : ETAT DE L'ART
%%%%%%%%%%%%%%%%%%%%%%%%%%%%%%%

En Intelligence Artificielle, la \emph{planification} est un processus cognitif qui consiste \`a g\'en\'erer automatiquement, au travers d'une proc\'edure formelle, un r\'esultat articul\'e sous la forme d'un syst\`eme de d\'ecision int\'egr\'e appel\'e \emph{plan}. Le plan est g\'en\'eralement sous la forme d'une collection organis\'ee d'\emph{actions} et il doit permettre \`a l'univers d'\'evoluer de l'\emph{\'etat initial} \`a un \'etat qui satisfait le \emph{but}. Dans le cadre classique, le plus restrictif, on consid\`{e}re les actions comme des transitions instantan\'{e}es sans prendre en compte le temps.

\fred{Intro planif. temporelle}

%La planification par satisfaction de base de clauses (planification SAT) a \'et\'e introduite avec le planificateur SATPLAN \cite{kautzS92_planning_sat}. Dans cette approche, on travaille directement sur un ensemble fini de variables propositionnelles. Deux actions identiques pouvant appara\^{i}tre \`a des endroits diff\'erents d'un m\^{e}me plan doivent pouvoir \^{e}tre diff\'{e}renci\'{e}es et on leur associe donc des propositions diff\'erentes. Comme on ne conna\^{i}t pas \`a l'avance la longueur d'un plan solution d'un probl\`eme, on ne peut pas cr\'eer un codage unique permettant de le r\'esoudre puisqu'il faudrait cr\'eer une infinit\'e de variables propositionnelles pour repr\'esenter toutes les actions de tous les plans possibles. La solution la plus commune consiste alors \`a cr\'eer un codage repr\'esentant tous les plans d'une longueur $k$ fix\'ee. La base de clause ainsi obtenue est donn\'ee en entr\'ee \`a un solveur SAT qui retourne, lorsqu'il existe, un mod\`ele de cette base. Le d\'ecodage de ce mod\`ele permet alors d'obtenir un plan-solution. Si la r\'esolution du codage ne donne pas de mod\`ele, la valeur de $k$ est augment\'ee et le processus r\'eit\'er\'e. Pour la compl\'etude du proc\'ed\'e, tous ces mod\`eles doivent correspondre exactement \`a tous les plans solutions d'une longueur fix\'ee du probl\`eme.

\fred{bla bla planification générale (approches espaces d'états, espaces de plans, graphplan)}

Une des approches algorithmiques pour la synthèse de plans est la compilation automatique (c'est-à-dire, la transformation) de problèmes de planification. Dans le planificateur \textsc{Satplan}~\cite{KS92}, un problème de planification est compilé en une formule propositionnelle dont les modèles, correspondant aux plans-solutions, peuvent être trouvés en utilisant un solveur SAT.
L'approche SAT recherche un plan solution de longueur fixe $k$. En cas d'échec cette longueur est augmentée avant de relancer la recherche d'une solution. Dans le cadre classique, décider s'il existe une solution est PSPACE-complet, mais la décision de l'existence d'une solution de taille polynomiale par rapport à la taille du problème est NP-complete \cite{DBLP:journals/ai/Bylander94}.
Cette approche par compilation bénéficie directement des améliorations des solveurs SAT\footnote{\url{http://www.satcompetition.org/}}. L'exemple le plus marquant est le planificateur \textsc{Blackbox}~\cite{KS98a,KS99} (et ses successeurs \textsc{Satplan}'04~\cite{KAU04} et \textsc{Satplan}'06~\cite{KSH06}). Ces planificateurs ont obtenu la première place dans la catégorie planificateur optimal (en termes de nombre d'étapes du plan) des compétitions internationales de planification\footnote{\url{http://www.icaps-conference.org/index.php/Main/Competitions}} IPC-2004 et IPC-2006. Ce résultat était inattendu car ces planificateurs étaient essentiellement des mises à jour de \textsc{Blackbox} et n'incluaient aucune réelle nouveauté: l'amélioration des performances était principalement due aux progrès du solveur SAT sous-jacent.


%In the temporal framework, the complexity is EXPSPACE-complete in the general case, but it is PSPACE-complete when searching for a fixed-size solution \cite{DBLP:conf/aips/Rintanen07}.\\
%Despite the constant progress of SAT solvers, SAT planners have been largely supplanted in terms of resolution time by heuristic search planners. The winners of IPC-2014 are IBaCoP \cite{DBLP:journals/jair/CenamorRF16} (sequential-satisfycing track) which is based on a portfolio approach that define different ways to combine simple base heuristic planner, and SymBA* \cite{DBLP:journals/ai/TorralbaAKE17} (sequential-optimal track) also based on heuristic search.
%}

De nombreuses améliorations de cette approche originale ont été proposées depuis lors, notamment via le développement de codages plus compacts et plus efficaces~\cite{KS96,EMW97,MK98a,MK99,RIN03,RHN04,RHN06,DBLP:conf/aips/2008}. % mettre les références plus récentes
Suite à ces travaux, de nombreuses autres techniques similaires pour le codage de problèmes de planification ont été développées: Programmation Linéaire (LP) \cite{DBLP:conf/ijcai/WolfmanW99}, Problèmes de Satisfaction de Contraintes (CSP) \cite{DBLP:journals/ai/DoK01}, SAT Modulo Theories (SMT) \cite{DBLP:journals/ai/ShinD05,DBLP:conf/ictai/MarisR08,DBLP:conf/aaai/Rintanen15}. Plus récemment, une approche QBF (Quantified Boolean Formulas) a été proposée par \cite{DBLP:conf/aaai/Rintanen07,DBLP:conf/ecai/CashmoreFG12}.


Actuellement, les solveurs SAT surpassent les solveurs QBF et l'approche SAT est la plus efficace car les solveurs et les codages ont été grandement améliorés depuis 1992. Cependant, au cours de la dernière décennie, l'approche QBF a suscité un intérêt croissant. L'évaluation compétitive QBFEVAL\footnote{\url{http://www.qbflib.org/index_eval.php}} des solveurs QBF est désormais un événement lié à la conférence internationale SAT et les solveurs QBF s'améliorent régulièrement. QBFEVAL'16 comptait plus de participants que jamais et les articles relatifs à QBF y représentaient 27\% de tous les articles publiés à SAT'16. % (faire pour 2017, Fred: il y a un bien moins en 2017, il vaut mieux garder 2016).
Certaines techniques prometteuses telles que le raffinement d'abstraction guidé par contre-exemple (CEGAR)~\cite{DBLP:journals/jacm/ClarkeGJLV03,DBLP:conf/ijcai/JanotaM15,DBLP:journals/ai/JanotaKMC16,DBLP:conf/fmcad/RabeT15} ont été adaptées à la résolution QBF.
Pour des codages SAT / QBF comparables, l'approche QBF présente également l'avantage de générer des formules plus compactes \cite{DBLP:conf/ecai/CashmoreFG12}.
Actuellement, même si l'approche QBF n'est pas aussi efficace que l'approche SAT, elle mérite l'intérêt de la communauté.

\section{Présentation de la thèse}
