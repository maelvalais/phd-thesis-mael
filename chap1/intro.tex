\section{Introduction générale au domaine}

%\fred{La compilation de problèmes en logique bla bla résolution avec des solveurs bla bla}

L'utilisation de la logique pour résoudre des problèmes n'est pas neuve. Dans l'antiquité, Aristote utilisait des syllogismes pour raisonner et déduire des conclusions à partir de prémisses. La syllogistique perdurera jusqu'au Moyen Age. La "Dissertatio de arte combinatoria" publiée en 1666 par Gottfried Leibniz introduit l'idée novatrice de l’art de dériver des vérités de manière calculatoire, basé sur une \emph{characteristica universalis}, un langage mathématique non-ambigu et un \emph{calculus ratiocinator}, un calcul ou une machine manipulant la \emph{characteristica}.
En 1854, George Boole publie "An Investigation of the Laws of Thought on Which are Founded the Mathematical Theories of Logic and Probabilities". Dans son ouvrage, il propose un traitement algébrique, ainsi qu'une procédure de décision pour la logique propositionnelle.
En 1879, Gottlob Frege, que l'on peut considérer comme le fondateur de la logique moderne, publie le "Begriffsschrift". Il introduit les connecteurs essentiels de la logique des propositions $\rightarrow$ et $\neg$, les quantificateurs de la logique des prédicats et un calcul formel. Il fait la distinction entre une formule, qui représente une proposition
(qui peut être vraie ou fausse), et un jugement, qui est une formule dont on constate la vérité (dans un calcul donné).
Se basant sur ces travaux, les "Principia Mathematica", publiés en 1910-1913 par Alfred N. Whitehead et Bertrand Russell ont pour but de formaliser les mathématiques avec quelques notions élémentaires. Cependant, Kurt Gödel démontre en 1931 ses théorèmes d'incomplétude dans l'article "Sur les propositions formellement indécidables des Principia Mathematica et des systèmes apparentés". Toutefois, il prouve également la complétude de la logique du premier ordre.

En 1936, la thèse de Curch-Turing formalise la notion de calculabilité et permet d'envisager la conception des premiers ordinateurs Turing-complets. L'architecture d'un ordinateur entièrement électronique est élaborée par John von Neumann en juin 1945 dans le cadre du projet EDVAC (Electronic Discrete Variable Automatic Computer).
Lors de la conférence fondatrice de l'intelligence artificielle au Dartmouth College en 1956,  Allen Newell, Herbert A. Simon et Cliff Shaw présentent le premier prouveur automatique de théorèmes, le "Logic Theorist".

En 1960, Martin Davis et Hilary Putnam développent un algorithme basé sur le principe de résolution qui permet de déterminer la satisfiabilité d'une formule propositionnelle en forme normale conjonctive. En 1962, Martin Davis, Hilary Putnam, George Logemann et Donald Loveland proposent une extension appelée DPLL. C'est une procédure très efficace dont des implémentation modernes sont toujours aujourd'hui, 50 ans après, à la base d'une large majorité des solveurs SAT, QBF ou SMT (SAT Modulo Theories).
La performance des ces solveurs modernes nous permet aujourd'hui de les utiliser pour résoudre efficacement des problèmes modélisés par des formules logiques.

\section{Le cadre de travail}

\subsection{Les solveurs pour les logiques classiques}

\subsubsection{SAT : satisfaction de formules booléennes}

Le problème de satisfiabilité de formules booléennes (SAT) est connu comme étant le problème de référence pour la classe de complexité NP \cite{DBLP:conf/stoc/Cook71}.
Etant donné une formule propositionnelle en forme normale conjonctive (FNC), le problème SAT consiste à déterminer s'il existe un modèle de cette formule, c'est à dire une valuation pour chacune des variables de la formule.

Par exemple, si nous prenons un ensemble de variables propositionnelles $\{a,b,c\}$ et la formule $\Phi = (a \vee b) \wedge (\neg a \vee b \vee c) \wedge (\neg a \vee \neg b \vee \neg c)$, nous pouvons remarquer que $\Phi$ est satisfiable. En effet, il suffit que les variables $a$ et $b$ aient la valeur vrai ($\top$) et la variable $c$ ait la valeur faux ($\bot$), ce qui nous donne un modèle de $\Phi$.

Il est possible, comme nous le verrons dans le chapitre~\ref{chap:touist}, d'encoder de nombreux problèmes pour se ramener, pour les résoudre, à un problème de satisfiabilité de formule propositionnelle.

\subsubsection{SMT : SAT Modulo Theories}
%\subsubsection{SMT : SAT Modulo Theories}

Certains probl\`emes combinatoires n\'ecessitent n\'eanmoins de traiter des calculs sur les nombre naturels ou r\'eels. Ceci peut \^etre fait en utilisant seulement la logique propositionnelle (par exemple, $2+3=5$ pourrait \^etre cod\'e par $add_{2,3,5}$), mais c'est tr\`es inconfortable \`a moins qu'il n'y ait que quelques additions \`a faire. Ne parlons m\^eme pas des op\'erations de multiplication ou plus complexes. L'id\'ee derri\`ere la gen\`ese de SMT a \'et\'e de combiner des solveurs SAT avec un solveur arithm\'etique dans le but d'am\'eliorer le traitement de la partie arithm\'etique du raisonnement. Dans de nombreux cas, ceci n'am\'eliorera pas seulement l'efficacit\'e du solveur, mais permettra aussi d'exprimer les contraintes arithm\'etiques des probl\`emes d'une mani\`ere radicalement plus compacte.

Pensez au jeu de Kamaji\footnote{\texttt{http://fr.wikipedia.org/wiki/Kamaji}}  o\`u le joueur doit grouper des nombres adjacents dans une grille de sorte que leur somme soit \'egale \`a un nombre fixe. R\'esoudre le jeu n\'ecessite essentiellement un raisonnement logique mais a aussi besoin d'un peu d'arithm\'etique (addition).


Si $x_{i,j}$ pour chaque case $(i,j)$ est un entier et $G(i,j,i,k)$ repr\'esente le fait que les cases $(i,j)$ \`a $(i,k)$ de la ligne $i$ forment un groupe, la contrainte de somme pourrait \^etre exprim\'ee par :
$$\sum_{m\in E}x_{i,m}=N$$
o\`u $N$ est le nombre fixe et $E$ est $\{j,j+1,\ldots,k\}$. La logique propositionnelle pure n'est certainement pas adapt\'ee pour de telles phrases !

\subsubsection{QBF : formules booléennes quantifiées}
Quantified Boolean Formula (QBF) est connu comme étant le problème de référence pour la classe de complexité PSPACE (\cite{Stockmeyer:1973:WPR:800125.804029}). C'est une extension de la logique propositionnelle permettant de quantifier sur les variables propositionnelles. 

Par exemple, $\forall p \exists q.p \leftrightarrow q$ se lit : pour toute valeur de vérité de $p$, il existe une valeur de vérité de $q$ tel que $p \leftrightarrow q$ est vrai. Cette formule est vraie (il suffit de choisir la même valeur pour $q$ que pour $p$). Alors que  $\exists p \forall q.p \lor q$ ne l'est pas. Ainsi, une formule booléenne quantifiée est toujours SOIT vraie SOIT fausse. 

De fait, à toute formule QBF peut être associée une formule propositionnelle sans variables car par définition : 
$\forall p.\Phi$ est vraie ssi $\Phi_{[p:=\top]} \wedge \Phi_{[p:=\bot]}$ l'est, et $\exists p.\Phi$ est vraie ssi $\Phi_{[p:=\top]} \vee \Phi_{[p:=\bot]}$. 
La formule QBF peut être exponentiellement plus compacte que la formule propositionnelle correspondante. Par exemple à la formule $\forall p \exists q.p \leftrightarrow q$ correspond la formule propositionnelle %\\
$(%\Big ( 
(\top \leftrightarrow \top)\vee (\top \leftrightarrow \bot) )%\Big ) 
\wedge (%\Big ( 
(\bot \leftrightarrow \top)\vee (\bot \leftrightarrow \bot) )%\Big )
$.

\subsection{La planification par compilation automatique}
%%%%%%%%%%%%%%%%%%%%%%%%%%%%%%%
% PLANIFICATION : ETAT DE L'ART
%%%%%%%%%%%%%%%%%%%%%%%%%%%%%%%

En Intelligence Artificielle, la \emph{planification} est un processus cognitif qui consiste à générer automatiquement, au travers d'une procédure formelle, un résultat articulé sous la forme d'un système de décision intégré appelé \emph{plan}. Le plan est généralement sous la forme d'une collection organisée d'\emph{actions} et il doit permettre à l'univers d'évoluer de l'\emph{état initial} à un état qui satisfait le \emph{but}. Dans le cadre classique, le plus restrictif, on considère les actions comme des transitions instantanées sans prendre en compte le temps.
Dans le cadre temporel, on considère que les actions ont une durée d'exécution et que les événements associés aux actions ne sont plus instantanés mais peuvent avoir lieu à différents instants liés par des contraintes inhérentes aux actions.

%La planification par satisfaction de base de clauses (planification SAT) a été introduite avec le planificateur SATPLAN \cite{kautzS92_planning_sat}. Dans cette approche, on travaille directement sur un ensemble fini de variables propositionnelles. Deux actions identiques pouvant apparaître à des endroits différents d'un même plan doivent pouvoir être différenciées et on leur associe donc des propositions différentes. Comme on ne connaît pas à l'avance la longueur d'un plan solution d'un problème, on ne peut pas créer un codage unique permettant de le résoudre puisqu'il faudrait créer une infinité de variables propositionnelles pour représenter toutes les actions de tous les plans possibles. La solution la plus commune consiste alors à créer un codage représentant tous les plans d'une longueur $k$ fixée. La base de clause ainsi obtenue est donnée en entrée à un solveur SAT qui retourne, lorsqu'il existe, un modèle de cette base. Le décodage de ce modèle permet alors d'obtenir un plan-solution. Si la résolution du codage ne donne pas de modèle, la valeur de $k$ est augmentée et le processus réitéré. Pour la complétude du procédé, tous ces modèles doivent correspondre exactement à tous les plans solutions d'une longueur fixée du problème.

%\fred{bla bla planification générale (approches espaces d'états, espaces de plans, graphplan)}

Une des approches algorithmiques pour la synthèse de plans est la compilation automatique (c'est-à-dire, la transformation) de problèmes de planification. Dans le planificateur \textsc{Satplan}~\cite{KS92}, un problème de planification est compilé en une formule propositionnelle dont les modèles, correspondant aux plans-solutions, peuvent être trouvés en utilisant un solveur SAT.
L'approche SAT recherche un plan solution de longueur fixe $k$. En cas d'échec cette longueur est augmentée avant de relancer la recherche d'une solution. Dans le cadre classique, décider s'il existe une solution est PSPACE-complet, mais la décision de l'existence d'une solution de taille polynomiale par rapport à la taille du problème est NP-complete \cite{DBLP:journals/ai/Bylander94}.
Cette approche par compilation bénéficie directement des améliorations des solveurs SAT\footnote{\url{http://www.satcompetition.org/}}. L'exemple le plus marquant est le planificateur \textsc{Blackbox}~\cite{KS98a,KS99} (et ses successeurs \textsc{Satplan}'04~\cite{KAU04} et \textsc{Satplan}'06~\cite{KSH06}). Ces planificateurs ont obtenu la première place dans la catégorie planificateur optimal (en termes de nombre d'étapes du plan) des compétitions internationales de planification\footnote{\url{http://www.icaps-conference.org/index.php/Main/Competitions}} IPC-2004 et IPC-2006. Ce résultat était inattendu car ces planificateurs étaient essentiellement des mises à jour de \textsc{Blackbox} et n'incluaient aucune réelle nouveauté: l'amélioration des performances était principalement due aux progrès du solveur SAT sous-jacent.


%In the temporal framework, the complexity is EXPSPACE-complete in the general case, but it is PSPACE-complete when searching for a fixed-size solution \cite{DBLP:conf/aips/Rintanen07}.\\
%Despite the constant progress of SAT solvers, SAT planners have been largely supplanted in terms of resolution time by heuristic search planners. The winners of IPC-2014 are IBaCoP \cite{DBLP:journals/jair/CenamorRF16} (sequential-satisfycing track) which is based on a portfolio approach that define different ways to combine simple base heuristic planner, and SymBA* \cite{DBLP:journals/ai/TorralbaAKE17} (sequential-optimal track) also based on heuristic search.
%}

De nombreuses améliorations de cette approche originale ont été proposées depuis lors, notamment via le développement de codages plus compacts et plus efficaces~\cite{KS96,EMW97,MK98a,MK99,RIN03,RHN04,RHN06,DBLP:conf/aips/2008}. % mettre les références plus récentes
Suite à ces travaux, de nombreuses autres techniques similaires pour le codage de problèmes de planification ont été développées: Programmation Linéaire (LP) \cite{DBLP:conf/ijcai/WolfmanW99}, Problèmes de Satisfaction de Contraintes (CSP) \cite{DBLP:journals/ai/DoK01}, SAT Modulo Theories (SMT) \cite{DBLP:journals/ai/ShinD05,DBLP:conf/ictai/MarisR08,DBLP:conf/aaai/Rintanen15}. Plus récemment, une approche QBF (Quantified Boolean Formulas) a été proposée par \cite{DBLP:conf/aaai/Rintanen07,DBLP:conf/ecai/CashmoreFG12} et nous avons proposé de nouveaux codages QBF dans \cite{GasquetLMRV18}.

Actuellement, les solveurs SAT surpassent les solveurs QBF et l'approche SAT est la plus efficace car les solveurs et les codages ont été grandement améliorés depuis 1992. Cependant, au cours de la dernière décennie, l'approche QBF a suscité un intérêt croissant. L'évaluation compétitive QBFEVAL\footnote{\url{http://www.qbflib.org/index_eval.php}} des solveurs QBF est désormais un événement lié à la conférence internationale SAT et les solveurs QBF s'améliorent régulièrement. QBFEVAL'16 comptait plus de participants que jamais et les articles relatifs à QBF y représentaient 27\% de tous les articles publiés à SAT'16. % (faire pour 2017, Fred: il y a un bien moins en 2017, il vaut mieux garder 2016).
Certaines techniques prometteuses telles que le raffinement d'abstraction guidé par contre-exemple (CEGAR)~\cite{DBLP:journals/jacm/ClarkeGJLV03,DBLP:conf/ijcai/JanotaM15,DBLP:journals/ai/JanotaKMC16,DBLP:conf/fmcad/RabeT15} ont été adaptées à la résolution QBF.
Pour des codages SAT / QBF comparables, l'approche QBF présente également l'avantage de générer des formules plus compactes \cite{DBLP:conf/ecai/CashmoreFG12}.
Actuellement, même si l'approche QBF n'est pas aussi efficace que l'approche SAT, elle mérite l'intérêt de la communauté.

\section{Présentation de la thèse}

Dans cette thèse, nous nous sommes intéressés aux codages de problèmes en logique et tout particulièrement aux codages et à la résolution automatique de problèmes de planification en utilisant des solveurs.

Dans le chapitre~\ref{chap:touist}, nous présentons notre traducteur automatique \touist qui permet de coder des problèmes en formules logiques et de les résoudre en utilisant un solveur SAT, QBF ou SMT.
Dans la section~\ref{chap:touist:historique}, nous présentons un historique et une vue d'ensemble de \touist. Ensuite, nous introduisons les différentes fonctionnalités de notre traducteur dans la section~\ref{chap:touist:touist}. Nous introduisons tout d'abord dans la sous-section~\ref{chap:touist:touist:sat-statique} comment modéliser des problèmes combinatoires statiques, comme le Sudoku, avec SAT. Ensuite, dans la sous-section~\ref{chap:touist:touist:sat-dynamique} nous montrons comment prendre en compte des aspects dynamiques avec une modélisation du jeu de Nim avec SAT. Nous montrons alors, dans la sous-section~\ref{chap:touist:touist:qbf}, comment trouver une stratégie gagnante pour le jeu de Nim en utilisant QBF. Ensuite, dans la sous-section~\ref{chap:touist:touist:smt}, nous donnons un exemple de modélisation du jeu de Takuzu en utilisant SMT avec des contraintes linéaires.
%Nous présentons tout d'abord les langages logiques que nous allons utiliser avant d'illustrer leur utilisation au travers d'un exemple de codage du jeu de Nim.
Nous présentons enfin, dans la section~\ref{chap:touist:touistplan}, le module \touistplan que nous avons implémenté pour résoudre automatiquement des problèmes de planification en utilisant les codages que nous détaillons dans le chapitre~\ref{chap:codages} avec notre traducteur \touist.


Dans le chapitre~\ref{chap:codages}, nous présentons différents codages de problèmes de planification en logique. Dans la section~\ref{chap:codages:satqbf}, nous nous plaçons dans un premier temps dans le cadre classique de la planification. Nous présentons des codages SAT de référence dans les espaces d'états \cite{KS92,KS95} et dans les espaces de plans \cite{MK99} avant d'introduire un nouveau codage dans les espaces de plans basé sur le découpage des liens causaux en modélisant la notion de condition ouverte. Nous présentons ensuite un codage QBF de référence basé sur l'utilisation d'une représentation d'arbre compact (CTE) et des actions No-ops comme frame-axiomes \cite{DBLP:conf/ecai/CashmoreFG12,DBLP:phd/ethos/Cashmore13}. Nous introduisons alors deux nouveaux codages QBF compacts inspirés par les codages SAT précédemment introduits. Dans la section~\ref{chap:codages:smt}, nous nous plaçons dans le cadre étendu de la planification temporelle et détaillons un codage SMT de référence \cite{DBLP:conf/ictai/MarisR08} avant d'en présenter une adaptation plus compacte basée sur le découpage des liens causaux que nous avons proposé pour la planification classique.
Dans la section~\ref{chap:codages:tests}, après avoir présenté les problèmes de planification de référence issus de différentes compétitions internationales de planification (IPC), nous comparons à l'aide du module \touistplan les performances du codage QBF de référence présenté dans la sous-section~\ref{chap:codages:qbf:reference} et de nos nouveaux codages introduits dans la sous-section~\ref{chap:codages:qbf:nouveaux}.
Nous montrons ainsi que nos codages sont deux fois plus efficaces en terme de temps de résolution.
%Puis, nous  Après avoir présenté le module \touistplan et les problèmes de planification de référence issus de différentes compétitions internationales de planification (IPC), nous montrons que nos codages sont plus efficace en terme de temps de résolution.



%\fred{A revoir:}
%Nous allons montrer que pour l'approche QBF, au-delà des améliorations sur les solveurs, d'autres travaux doivent être menés pour mettre au point des codages plus performants. Comme nous l'avons indiqué plus haut, cela a été réalisé pour SAT avec des améliorations significatives. Nous présentons ici deux nouveaux codages d'arbres compacts CTE (Compact Tree Encoding) de problèmes de planification en QBF: CTE-EFA basé sur des frame-axiomes explicatifs dans un espace d'états, et CTE-OPEN basé sur des liens de causalité dans un espace de plans. \fred{dans le chapitre EXPERIMENTATIONS:} Nous les comparons au codage de l'état de l'art CTE-NOOP basé sur des actions No-op et proposé dans \cite{DBLP:conf/ecai/CashmoreFG12}. En termes de temps d'exécution par rapport aux problèmes de référence, CTE-EFA et CTE-OPEN sont toujours plus efficaces que CTE-NOOP.